\section{Numerical Experiments}
In this section, the LP and SDP relaxations as well as rounding schemes introduced in sections \ref{sec:lpRelax} and \ref{sec:sdpRelax} are put to the test for key CSP problems, such as Max-Cut and Max k-SAT. This section provides empirical validation and identifies their usefulness in practical applications. In particular, this section contains numerical studies for the LP relaxation and the Max k-SAT rounding scheme and the SDP relaxation in combination with the Goeman-Williamson rounding scheme for Max Cut. 

\subsection{LP Relaxation and Rounding scheme for Max k-SAT}
The LP relaxation as introduced in section \ref{sec:lpRelax} can be applied to any CSP. However, it does not provide a feasible solution to the CSP ( unless the optimal solution happens to be feasible for CSP) since the LP is a relaxation of a CSP. Therefore, its application is dependent on a technique to generate feasible solutions from the LP solution. Such a technique was introduced for K-SAT in section \ref{sec:lpRoundingSat}. In this section, results are presented of applying the LP relaxation and the rounding scheme to Max 3-SAT on randomly generated 3-SAT instances using CVX.

The 3-SAT instances where generated as follows:
\begin{enumerate}
	\item A solution $x^*$ on $k$ variables is generated. Each variable is independently set to $1$ or $0$, where $1$ is chosen with probability $p_T$. $p_T$ is the probability of setting a variable to true. This is probability is drawn from a uniform distribution on $[0,1]$. 
	\item $m$ constraints are added. These constraints are constructed via the following procedure:
	\begin{enumerate}
		\item The number of negated variables in the scope is uniformly drawn from $[0,1,2,3]$.
		\item For the scope, 3 variables are randomly drawn with replacement.
		\item A constraint is accepted if the feasible solution satisfies the constraint and rejected otherwise.
	\end{enumerate}
\end{enumerate}

\begin{table}
	\footnotesize
	\centering
	\caption{Experimental results for LP relaxation and SAT rounding. Values reported are averages of 50 randomly generated instances. $z^*$ is the }
	\begin{tabularx}{\textwidth}{>{\centering}p{1.7cm}>{\centering}p{1.7cm}ssCCmC}
		\toprule
		Number of variables & Number of constraints & Mean $z^*$ (s.d.) & Mean $z^*_{LP}$ (s.d.) & Mean $z^*_{round}$ (s.d.) & Mean $t_{gen}$ (s.d.) & Mean $t_{solve}$ (s.d.) & Mean $t_{round}$ (s.d.) \\ \midrule
		10  & 20  & 1 (0) & 1 (0) & 0.89 (0.08) & 0.032 (0.006) & 0.78 (0.03) & 0.00 (0.00) \\
		10  & 50  & 1 (0) & 1 (0) & 0.89 (0.05) & 0.08 (0.01)   & 1.51 (0.04) & 0.00 (0.00) \\
		20  & 40  & 1 (0) & 1 (0) & 0.88 (0.05) & 0.066 (0.008) & 1.29 (0.05) & 0.00 (0.00) \\
		20  & 100 & 1 (0) & 1 (0) & 0.89 (0.04) & 0.19 (0.02)   & 2.8 (0.2)   & 0.00 (0.00) \\
		50  & 50  & 1 (0) & 1 (0) & 0.89 (0.04) & 0.08 (0.01)   & 1.58 (0.05) & 0.00 (0.00) \\
		50  & 100 & 1 (0) & 1 (0) & 0.89 (0.03) & 0.19 (0.03)   & 2.81 (0.06) & 0.00 (0.00) \\
		50  & 200 & 1 (0) & 1 (0) & 0.88 (0.02) & 0.50 (0.08)   & 5.5 (0.3)   & 0.00 (0.00) \\
		100 & 100 & 1 (0) & 1 (0) & 0.90 (0.04) & 0.20 (0.03)   & 3.0 (0.1)   & 0.00 (0.00) \\
		100 & 200 & 1 (0) & 1 (0) & 0.88 (0.02) & 0.6 (0.1)     & 7 (1)       & 0.00 (0.00) \\
		\bottomrule
	\end{tabularx}
	\label{tab:expSat}
\end{table}

The results of the experiments are reported in Table \ref{tab:expSat}. It is observed that generating the CVX formulation as well as solving the LP is the bottleneck. As expected, this increases when the problem instances become large. Furthermore, note that the proven bound of $0.63$ is indeed satisfied. This is an empirical verification of the stated theorem.

\subsection{Experiments with Max-Cut}