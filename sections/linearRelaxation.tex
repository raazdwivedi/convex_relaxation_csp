\section{LP relaxations for CSP}
The goal of this section is to develop a canonical LP relaxation for CSP problems. The relaxations in this section can be applied for any CSP. 
In addition, for some CSPs other LP relaxations have been developed. However, these relaxations are not within the scope of this project.

\subsection{Simple LP relaxation}
Probably the most intuitive integer program for CSPs has variables $z_c$ to denote whether constraint $c \in C$ is satisfied and binary variables $x_v^t$ to denote whether variable $v \in V$ takes value $t \in D$. 
Then, a CSP can be denoted as the following integer program.
\begin{alignat}{3}
	\max \quad & \sum_{c \in C}w_c z_c & \label{eq:lp_weak}\\
	\text{s.t.} \quad & z_c \le R_c( x_{S_c} ) & \quad c \in C \nonumber \\
	& \sum_{t \in D} x_v^t = 1 & \quad v \in V \nonumber\\
	& z_c \in \{0,1\} & \quad c \in C \nonumber\\z
	&	x_v^t \in \{0,1\}	& \quad v \in  V \nonumber
\end{alignat}
where $x_{S_c}$ are the variables corresponding to the CSP variables in the scope of constraint $c \in C$ and $R_c$ is a linear formulation of the operator $R_c \in \Gamma$. 
That is $R_c$ evaluates to $1$ if the constraint is satisfied and $0$ otherwise. 
Note that if the constraint $c$ is satisfied, then $z_c = 1$ and $w_c$ is contributed to the objective function, and if the constraint is not satisfied, then $z_c$ is forced to be zero. 

A convex relaxation can be obtain by relaxing the integrality constraints, and require that $x_v^t, z_c \in [ 0,1]$. 
However, the corresponding LP relaxation is usually weak. In particular, consider the case of Max-Cut. 
Then, following the formulation in \eqref{eq:lp_weak}, the Max-Cut problem can be formulated as:
\begin{alignat}{3}
\max \quad & \sum_{c \in C}w_{uv} z_{uv} & \label{eq:maxCut_weak}\\
\text{s.t.} \quad & z_{uv} \le x_u + x_v & \quad [u,v] \in E \label{eq:maxCut_low} \\
& z_{uv} \le 2 - ( x_u + x_v ) & \quad [u,v] \in E \label{eq:maxCut_upp}\\
& z_{uv} \in \{0,1\} & \quad [u,v] \in E \nonumber\\
&	x_v \in \{0,1\}	& \quad v \in  V \nonumber
\end{alignat}
where we replace $x_{u}^0$ and $x_u^1$ with a single variable $x_u$. 
Constraints \eqref{eq:maxCut_low} and \eqref{eq:maxCut_upp} ensure that $w_{uv}$ contributes to the cut if and only if $x_u$ and $x_v$ have different values.

Let us consider the LP formulation for the formulation given in \eqref{eq:maxCut_weak}. 
By the definition of CSP, the weights sum up to $1$. 
Hence, the maximum achievable objective value for the LP relaxation is $1$. 
Consider the solution $z_{uv}$ equal to $1$ for all $[u,v] \in E$ (i.e. the set of constraints $C$) and $x_v = \frac{1}{2}$. 
Observe that this solution is feasible for the LP relaxation, since each of the constraints. Furthermore, the solution is optimal since it achieves the maximum achievable value of $1$. 
However, this LP relaxation does not provide information about the best value of the variable $x_u$. 
Therefore, a more involved LP formulation is considered.

\subsection{Canonical LP relaxation}