\section{LP relaxations for CSP}
The goal of this section is to develop a canonical LP relaxation for CSP problems. The relaxations in this section can be applied for any CSP. 
In addition, for some CSPs other LP relaxations have been developed. However, these relaxations are not within the scope of this project.

\subsection{Simple LP relaxation}
Probably the most intuitive integer program for CSPs has variables $z_c$ to denote whether constraint $c \in \mathcal{C}$ is satisfied and binary variables $\mu_v^t$ to denote whether variable $v \in V$ takes value $t \in D$. 
Then, a CSP can be denoted as the following integer program.
\begin{alignat}{3}
	\max \quad & \sum_{C_i \in \mathcal{C}}w_{C_i} z_{C_i} & \label{eq:lp_weak}\\
	\text{s.t.} \quad & z_{C_i} \le R_{i}( \mu_{S_i} ) & \quad C_i = (R_i, S_i) \in \mathcal{C} \nonumber \\
	& \sum_{t \in D} \mu_v^t = 1 & \quad v \in V \nonumber\\
	& z_{C_i} \in \{0,1\} & \quad {C_i} \in \mathcal{C} \nonumber\\
	&	\mu_v^t \in \{0,1\}	& \quad v \in  V \nonumber
\end{alignat}
where $\mu_{S_i}$ are the variables corresponding to the CSP variables $x_{S_i}$ in the scope of constraint ${C_i} \in \mathcal{C}$ and $R_i$ is a linear formulation\footnote{Assuming that such a formulation is indeed possible.} of the operator $R_i \in \Gamma$ corresponding to constraint $C_i$. 
That is $R_i$ evaluates to $1$ if the constraint is satisfied and $0$ otherwise. 
Note that if the constraint ${C_i}$ is satisfied, then $z_{C_i} = 1$ and $w_{C_i}$ is contributed to the objective function, and if the constraint is not satisfied, then $z_{C_i}$ is forced to be zero. 

A convex relaxation can be obtain by relaxing the integrality constraints, and require that $\mu_v^t, z_{C_i} \in [ 0,1]$. 
Then, all $\mu_v^t$ for $t \in D$ can be considered as a probability distribution over the possible values $x_v$ can take.
However, this LP relaxation is usually weak, and it provides no information about possible assignments. In particular, consider the case of Max-Cut. 
Then, following the formulation in \eqref{eq:lp_weak}, the Max-Cut problem can be formulated as:
\begin{alignat}{3}
\max \quad & \sum_{[u,v]\in E}w_{uv} z_{uv} & \label{eq:maxCut_weak}\\
\text{s.t.} \quad & z_{uv} \le x_u + x_v & \quad [u,v] \in E \label{eq:maxCut_low} \\
& z_{uv} \le 2 - ( x_u + x_v ) & \quad [u,v] \in E \label{eq:maxCut_upp}\\
& z_{uv} \in \{0,1\} & \quad [u,v] \in E \nonumber\\
&	x_v \in \{0,1\}	& \quad v \in  V \nonumber
\end{alignat}
where we replace $\mu_{u}^0$ and $\mu_u^1$ with a single variable $x_u$. 
Constraints \eqref{eq:maxCut_low} and \eqref{eq:maxCut_upp} ensure that $w_{uv}$ contributes to the cut if and only if $x_u$ and $x_v$ have different values.

Let us consider the LP formulation for the formulation given in \eqref{eq:maxCut_weak}. 
By the definition of CSP, the weights sum up to $1$. 
Hence, the maximum achievable objective value for the LP relaxation is $1$. 
Consider the solution $z_{uv}$ equal to $1$ for all $[u,v] \in E$ (i.e. the set of constraints $C$) and $x_v = \frac{1}{2}$. 
Observe that this solution is feasible for the LP relaxation, since each of the constraints. Furthermore, the solution is optimal since it achieves the maximum achievable value of $1$. 
However, this LP relaxation does not provide information about the best value of the variable $x_u$. 
Therefore, a more involved LP formulation is considered.

\subsection{Canonical LP relaxation}
Although the above LP relaxation provides no information, LP relaxations of other integer programming formulations of CSP are a useful tool.
Let us consider the canonical LP relaxation used in the literature.
In addition to a probability distribution over the possible value each variable can take, the formulation contains a probability distribution over the possible assignments for each constraint.
This allows us to impose the constraint that the marginal probability for each variable matches the probability that it occurs in an assignment with the same value.

We use $\mu_v(\ell)$ to denote the probability that variable $x_v$ takes value $\ell \in D$.
Since these variables should form a probability distribution for a fixed $v \in V$, we require that the sum over $\ell \in D$ is equal to $1$ and that the $\mu_v(\ell)$ are non-negative.
Similarly, for constraint $C_i = (R_i, S_i) \in \mathcal{C}$ we introduce a probability distribution over all possible local assignments.
Let $\lambda_{C_i}$ be a probability distribution over  $\mathcal{L}_{C_i} = D^{|S_i|}$, all possible input values.
Then, the local assignment $L \in \mathcal{L}_{C_i}$ is selected with probability $\lambda_{C_i}[L]$.
For example, suppose $D = \{0,1\}$ and $S_i = (x_2,x_3)$. 
Then, the corresponding variables are $\lambda_{C_i}[x_2 \rightarrow 0, x_3 \rightarrow 0], \lambda_{C_i}[x_2 \rightarrow 0, x_3 \rightarrow 1], \lambda_{C_i}[x_2 \rightarrow 1, x_3 \rightarrow 0], \lambda_{C_i}[x_2 \rightarrow 1, x_3 \rightarrow 1]$. Again, we require that the variables corresponding to a given constraint $C_i \in \mathcal{C}$ sum up to 1 and are non-negative.

The objective is to maximize the probability, denoted as the expectation of an indicator, that the constraint $C_i$ is satisfied by a local assignment $\lambda_{C_i}[\cdot]$ where $C_i \in \mathcal{C}$ is randomly drawn from distribution $W$ and local assignment $L$ is randomly drawn with probability $\lambda_{C_i}[L]$. 
Mathematically, this can be written as:
\[
	OPT_{LP} = \max_{\lambda_{\mathcal{C}}} \underset{C_i \sim W}{\Ex} \left[ \underset{L \sim \lambda_{C_i}}{\Ex} \left[ R_i( L (S_i )) \right] \right]
\]
where $R_i( L (S_i ))$ is an indicator whether assignment $L$ for scope $S_i$ satisfies the relation given by $R_i$.

Without additional constraints, the current LP is not practical. 
The problem is that $\lambda_C$ variables are not coupled across constraints in $\mathcal{C}$. 
This allows us to separately select a satisfying assignment for each of the constraints. 
Hence, $OPT_{LP} = 1$ regardless of the particular instance. 
As eluded upon, this can be addressed by coupling the marginal probabilities for a variable $x_v$ based on the assignment probabilities $\lambda_C$ variables with $x_v$ in its scope and the corresponding $\mu_v$ variables.
For each constraint $C_i \in \mathcal{C}$, each variable $v \in V$, and for each $ \ell \in D$, we require that:
\[
	\underset{L \sim \lambda_{C_i} }{\mathbb{P}}[ L(v) = \ell] = \underset{L \sim \mu_v }{\mathbb{P}}[ L = \ell]
\]
We refer to these constraints as first-order consistency constraints.
It remains to show that these constraints are indeed linear.
If $T_{C_i}(v)$ is the set of local assignments for constraint $C_i \in \mathcal{C}$ that assigns variable $x_v$  the value $\ell \in D$, then the corresponding constraint can be written as:
\[
	\sum_{\substack{L \in \mathcal{L}_{C_i}\\ L(v) = \ell}} \lambda_{C_i}[L] = \mu_v(\ell)
\]
For the example given above with scope $(x_2,x_3)$, $D = \{0,1\}$, variable $v = 3$, and $ \ell = 1$, the constraint can be written as:
\[
	\lambda_{C_i}[x_2 \rightarrow 0, x_3 \rightarrow 1] + \lambda_{C_i}[x_2 \rightarrow 1, x_3 \rightarrow 1] = \mu_3^1
\]

The complete formulation is then:
\begin{alignat}{3}
\max \quad & \sum_{C_i \in \mathcal{C}} w_{C_i} \sum_{ L \in \mathcal{L}_{C_i} }  \lambda_{C_i}[L] \  R_i(L(S_i))) \label{eq:lp_strong}\\
\text{s.t.} \quad & 	\sum_{\substack{L \in \mathcal{L}_{C_i}\\ L(v) = \ell }} \lambda_{C_i}[L] = \mu_v(\ell) & \quad C_i \in \mathcal{C}, v \in V, \ell \in D \nonumber \\
& \sum_{L \in \mathcal{L}_{C_i}} \lambda_{C_i}[ L ] = 1 & \quad  C_i \in \mathcal{C} \nonumber\\
& \sum_{\ell \in D} \mu_v(\ell) = 1 & \quad v \in V \nonumber\\
& \lambda_{C_i}[ L ] \ge 0 & \quad C_i \in \mathcal{C}, L \in \mathcal{L}_{C_i} \nonumber\\
&	\mu_v(\ell) \ge 0	& \quad v \in  V, \ell \in D \nonumber
\end{alignat}
where $\mathcal{L}_{C_i}$ is the set of all possible local input assignments for the variables in scope $S_i$.

Observe that this formulation couples the local assignments for different constraints through the variables $\mu$. 
Stronger relaxations are possible by requiring that all pairwise marginals are preserved. For example, for all sets of two variables. 
However, this results in a semi-infinite LP with an exponential number of constraints. 

\begin{thm}
	The LP given in \eqref{eq:lp_strong} is a relaxation for the problem CSP$(\Gamma)$
\end{thm}
\begin{proof}
	Given an optimal solution for an instance of the problem CSP($\Gamma$), assign $\mu_v^t$ to be $1$ if the variable $x_v$ takes value $t$ in the optimal solution to the CSP($\Gamma$) instance and $0$ otherwise.
	Furthermore, for each $C_i \in \mathcal{C}$ set $\lambda_{C_i}[y]$ equal to $1$ if $y$ is the local assignment for the scope $S_i$ corresponding to the optimal solution to the CSP and $0$ otherwise. 
	It follows that this is a feasible LP solution. 
	Hence, $OPT(\mathcal{C}) \le OPT_{LP}(\mathcal{C})$.
\end{proof}

Having obtained a valid LP relaxation for any CSP problem, the question is now how to use this relaxation to generate good feasible solutions to the CSP. 
A key technique to generate CSP solutions is rounding the LP solution.

\subsection{Rounding schemes for LP relaxations}
Various techniques have been developed to extract good CSP solutions from the LP relaxation in \eqref{eq:lp_strong}\improvement{give references}.
A common technique to generate CSP solutions with a performance guarantee is randomized rounding.
This technique uses the information available from the LP, typically by using the optimal solution as a probability distribution, to randomly round the each variable to a value in its domain $D$.
To showcase a potential rounding technique, let us consider the problem of Max $k$-SAT.

\subsubsection{Randomized Rounding scheme for LP relaxation of Max k-SAT}
Consider the full assignment $F$ as a vector of random variables. For each variable $v \in V$, let the random variable be given by:
\[
	F(v) = \begin{cases}
	1 & \text{with probability } \mu^*_v(1)\\
	0 & \text{with probability } \mu^*_v(0)
	\end{cases}
\]
where $\mu^*_v(\ell)$ is the value of variable $\mu_v(\ell)$ in an optimal solution to the LP.
Since the sum over $\ell \in D = \{0,1\}$ of  $\mu_v(\ell)$ is $1$, and since $\mu_v(\ell)$ is non-negative, the above is a valid definition for a random variable. 
Furthermore, note that $F(1), \dots, F(n)$ is a feasible solution to the Max k-SAT problem.

The expected objective value of this randomized assignment is given by:
\[
	\underset{F}{\Ex}\left[ \underset{C_i \sim W}{\mathbb{P}}[ F \text{ satisfies } C_i] \right]
\]
Since we can swap the order of expectation (where $\mathbb{P}$ is an expectation over an indicator), this is equivalent to
\[
\underset{C_i \sim W}{\mathbb{E}}\left[ \underset{F}{\mathbb{P}}[ F \text{ satisfies } C_i] \right]
\]
This swap allows us to consider each $C_i \in \mathcal{C}$ separately. 
Therefore, a single constraint $C_i \in \mathcal{C}$ can be considered.

Observe that since the constraints are in disjunctive form, each literal needs to evaluate to false for the constraint to be false. 
In other words, there is a unique falsifying assignment $b_{C_i}$ that makes constraint $C_i$ false. 
For example, if $C_i = x_1 \vee x_2 \vee \bar{x}_3$, then the unique falsifying assignment is given by $x_1 = 0$, $x_2 = 0$, and $x_3 = 1$. 
Considering the definition of $F(v)$ and the independence of $F(i)$ and $F(j)$ for $i \neq j$, it then follows that 
\begin{equation}
		\underset{F}{\mathbb{P}}[ F \text{ satisfies } C_i] = 1 - \prod_{v \in S_i} \mu_v(b_{C_i}(v)) \label{eq:objectiveRounding}
\end{equation}
where $b_{C_i}(v)$ is the value of variable $v \in S_i$ in the falsifying assignment. In particular, the probability of not satisfying the constraint in the above example is $ 1- \mu_1^*(0) \mu_2^*(0) \mu_3^*(1)$.

Having identified the objective value of the randomized solution, the question is whether any guarantees can be made on the quality of the solution compared to the LP. This can be done, by relating $F$ to the distribution over local assignments.
\begin{align}
	p_{C_i} &:= \underset{L \sim \lambda_{C_i}^*}{\mathbb{P}} [L \text{ satisfies } C_i] \nonumber\\
			&=  \underset{L \sim \lambda_{C_i}^*}{\mathbb{P}}\left[ \bigcup_{v \in S} \left\{ L(v) \neq b_{C_i}(v) \right\} \right] \nonumber\\
			&\le \sum_{v \in  S } \underset{L \sim \lambda_{C_i}^*}{\mathbb{P}} \left[  \left\{ L(v) \neq b_{C_i}(v) \right\} \right] \nonumber\\
			&= \sum_{v \in S} \left( 1 - \mu_v^*(b_{C_i}(v)) \right) \label{eq:objectiveLP}
\end{align}
This last step follows from the first-order consistency constraints:
\[
	\underset{L \sim \lambda_{C_i} }{\mathbb{P}}[ L(v) = \ell] = \underset{L \sim \mu_v }{\mathbb{P}}[ L = \ell]
\]

It remains to relate the LP objective value to the expected objective of the rounding procedure. Note that the former (see \eqref{eq:objectiveLP}) is an arithmetic mean (AM) whereas the latter (see \eqref{eq:objectiveRounding}) is an geometric mean (GM). This suggests the use of the inequality of arithmetic and geometric means.
\begin{align*}
		\underset{F}{\mathbb{P}}[ F \text{ satisfies } C_i] &= 1 - \prod_{v \in S_i} \mu_v(b_{C_i}(v))\\
		&= 1 - \underset{v \in S}{GM}(\mu_{v}^*(b_c(v)))^{|S|}\\
		&\ge 1 - \underset{v \in S}{AM}(\mu_{v}^*(b_c(v)))^{|S|}\\
		&= 1 - \left( 1  - \underset{v \in S}{AM}(1 -\mu_{v}^*(b_c(v))) \right)^{|S|}\\
		&\ge 1 - \left( 1  - \frac{p_{C_i}}{|S|}\right)^{|S|}\\
\end{align*}

If $|S| \leftarrow \infty$, then the ratio approaches $1 - \frac{1}{e}$. Therefore, the objective value of the rounding scheme
\begin{align*}
	\underset{F}{\Ex} \left[ \text{Val}_{\mathcal{C}}[F]\right] &= \underset{F}{\Ex}\left[ \underset{C_i \sim W}{\mathbb{P}}[ F \text{ satisfies } C_i] \right]\\
	&=  \underset{C_i \sim W}{\mathbb{E}}\left[ \underset{F}{\mathbb{P}}[ F \text{ satisfies } C_i] \right]\\
	&\ge \left( 1 - \frac{1}{e} \right) \underset{C_i \sim W}{\Ex} [ p_{C_i} ]\\
	&= \left( 1 - \frac{1}{e} \right) \text{OPT}_{LP}(\mathcal{C})\\
	&\ge \left( 1 - \frac{1}{e} \right) \text{OPT}(\mathcal{C})\\
\end{align*}

Hence, we conclude with the following theorem:
\begin{thm}
	Randomized rounding is a $\left( ( 1 - 1/e)\beta, \beta \right)$-approximation for Max-SAT for any $\beta$.
\end{thm}
