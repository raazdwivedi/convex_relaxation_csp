\section{Linear Programming for CSP Approximation}\label{sec:lpRelax}
The goal of this section is to present a ``canonical" LP relaxation and rounding scheme for Constraint Satisfaction Problems. Both the relaxation and rounding scheme are valid for arbitrary CSP's, although performance guarantees are only proven in limited cases.  A non-exhaustive list of CSP-specific LP relaxations is included at the end of this section.

\subsection{An Integer Program : Towards the Canonical LP [CITE]}\label{subsec:ip}
Let $\mathcal{C} = (V,C,W)$ be a CSP over a domain $D$ of size $q$.

Earlier, we defined an \textit{assignment of variables} as a function $F : V \to D$. Critically, the domain of $F$ is the entire set $V$. When this is the case, we could call $F$ a \textit{full assignment}. It is reasonable (and as we will see, helpful!) to consider $F$ as being built from many \textit{local assignments} $L : S \to D$ where $S \subset V$. For a constraint $C_i = (R_i,S_i) \in \mathcal{C}$, we will be interested in the local assignment $L : S_i \to D$. Where before we could write $R_i(F(S_i))$ as the value of the constraint under an assignment, we write $R_i(L)$ when it is given that $L$ is a local assignment for constraint $C_i$.

Now consider an \textit{Integer}-Linear Program over the following variables:
\begin{itemize}
\item $\mu_v[\ell] \in \{0,1\}$ is an indicator that variable $v \in V$ takes value $\ell \in D$
\item $\lambda_i[L] \in  \{0,1\}$ is an indicator that \textit{local assignment} $L$ is used for constraint $C_i$
\end{itemize}
From these definitions, it is clear that for fixed $v$, we need one and only one $\mu_v[\ell]$ to be equal to 1. Encode this constraint as 
\begin{equation}\label{musum}
\sum_{\ell \in D} \mu_v[\ell] = 1.
\end{equation}

Now we define $\mathcal{L}_i$ as the set of all possible local assignments for the variables in constraint $C_i$'s scope. We note that the size of $\mathcal{L}_i$ is exponential in $t = ar(R_i)$ (in fact, it's exactly $|\mathcal{L}_i| = q^t$). This is one of the key reasons why maximum arity is a \textit{fixed parameter} for all $\mathcal{C} \in \text{CSP}(\Gamma)$.

Since any two local assignments $L_1$, $L_2$ in $ \mathcal{L}_i$ are mutually exclusive, we likewise need one and only one of $\lambda_i[L]$ equal to 1 for fixed $i$.
\begin{equation}\label{lambsum}
\sum_{L \in \mathcal{L}_i} \lambda_i[L] = 1 
\end{equation}\label{mulambcons}
To be consistent across $\lambda$ and $\mu$, we need one more constraint.
\begin{equation}
\mu_v[\ell] = \sum_{\substack{ L \in \mathcal{L}_i \\ L(v) = \ell }} \lambda_i[L] 
\end{equation}
Now we need to come up with an objective that mimics the one defined in Equation \ref{expfirst}. We re-write the expression for objective for clarity: 
$$\max_F \sum_{i:C_i \in C} w_i R_i(F(S_i))$$
As discussed in the beginning of this section it is easy to see 
\begin{equation}\label{replacewithlocal}
R_i(F(S_i)) = \sum_{L\in \mathcal L_i}\lambda_i[L] R_i(L).
\end{equation} 
It follows from the facts that whether or not $R_i$ is satisfied, depends only on the local assignment and RHS of \ref{replacewithlocal} is precisely these terms summed over all possible local assignments with multiplicative factor of indicators for the respective assignments. And thus our objective becomes 
\begin{equation}\label{glbobj}
\max_{\mu, \lambda} \sum_{i : C_i \in C} \sum_{L \in \mathcal{L}_i}   w_i\lambda_i[L] R_i(L).
\end{equation}

We get the corresponding LP simply by relaxing $\mu_v[\ell] \in \{0,1\}$ to $\mu_v[\ell] \in [0,1]$ and $\lambda_i[L] \in \{0,1\}$ to $\lambda_i[L] \in [0,1]$.


\subsection{The Canonical Linear Program}

We define the linear program below, then address its probabilistic interpretation and equivalent representations.
\begin{definition}\textbf{Basic LP} \\
Let $\mathcal{C} = (V,C,W)$ be a CSP over domain $D$. The Basic LP for $\mathcal{C}$ is
\begin{alignat}{2}
\max_{\mu, \lambda} ~&~ \sum_{i : C_i \in C} \sum_{L \in \mathcal{L}_i}   w_i\lambda_i[L] R_i(L) & \\
s.t. ~ & ~ \sum_{\ell \in D} \mu_v[\ell] = 1 & \forall v \in V  \label{eq:canonLPmuSum} \\
     ~ & ~ \sum_{L \in \mathcal{L}_i} \lambda_i[L] = 1  & \forall i : C_i \in C \label{eq:canonLPlambdaSum} \\
     ~ & ~ \sum_{\substack{ L \in \mathcal{L}_i \\ L(v) = \ell }} \lambda_i[L] = \mu_v[\ell]  & \forall v \in V, \ell \in D, i : C_i \in C \label{eq:canonLPConsistency} \\
     ~ & ~ 0 \leq \mu_v[\ell] \leq 1 & v \in V, \ell \in D \label{eq:canonLPmuNonNeg}\\
     ~ & ~ 0 \leq \lambda_i[L] \leq 1  & \forall  i : C_i \in C, L \in \mathcal{L}_i  \label{eq:canonLPlambdaNonNeg} 
\end{alignat}
\end{definition}

\begin{lemma}\label{le:super}
For the Basic LP, constraints \ref{eq:canonLPmuSum} and \ref{eq:canonLPmuNonNeg} can be dropped without altering optimal $\lambda^*$ or the optimal objective.
\end{lemma}
\begin{proof}
It is not difficult to see that constraints \ref{eq:canonLPmuSum} and \ref{eq:canonLPmuNonNeg} are implied by \ref{eq:canonLPlambdaSum}, \ref{eq:canonLPConsistency} and \ref{eq:canonLPlambdaNonNeg} :
\begin{align*}
\ref{eq:canonLPlambdaNonNeg} \rm{\ and \ } \ref{eq:canonLPlambdaSum} &\Rightarrow 0\leq  \sum_{\substack{ L \in \mathcal{L}_i \\ L(v) = \ell }} \lambda_i[L] \leq  \sum_{L \in \mathcal{L}_i} \lambda_i[L] = 1  \\
{\rm \ Combine\ with \ }\ref{eq:canonLPConsistency}&\Rightarrow  0\leq \mu_v[L] \leq 1
\end{align*}
Thus, \ref{eq:canonLPmuNonNeg} is implied. 

Also, summing on both sides of \ref{eq:canonLPConsistency} with respect to $\ell \in D$ we get, 
\begin{align*}
{\rm RHS} = \sum_{\ell \in D} \mu_v[\ell] = {\rm LHS} = \sum_{\ell \in D} \sum_{\substack{ L \in \mathcal{L}_i \\ L(v) = \ell }} \lambda_i[L] = \sum_{L \in \mathcal{L}_i} \lambda_i[L] \overset{\ref{eq:canonLPlambdaSum}}=1
\end{align*}
and thus \ref{eq:canonLPmuSum} is implied as well.
\end{proof}

In other words, the remark above renders $\mu_v[\cdot]$ indicators as superfluous. \emph{Nevertheless, we keep them for the discussion that follows, as they help in gaining some insight, help us come up with an intuitive rounding scheme and also lead us to extending the relaxation in a very natural fashion.}

\begin{thm}
	Basic LP is a relaxation for the problem CSP$(\Gamma)$.
\end{thm}
\begin{proof}
	Given an optimal solution for an instance of the problem CSP($\Gamma$), assign $\mu_v[t]$ to be $1$ if the variable $x_v$ takes value $t$ in the optimal solution to the CSP($\Gamma$) instance and $0$ otherwise.
	Furthermore, for each $C_i \in \mathcal{C}$ set $\lambda_{C_i}[y]$ equal to $1$ if $y$ is the local assignment for the scope $S_i$ corresponding to the optimal solution to the CSP and $0$ otherwise. 
	It follows that this is a feasible LP solution. 
	Hence, $OPT(\mathcal{C}) \le OPT_{LP}(\mathcal{C})$.
\end{proof}

\begin{remark}
It is worthwhile to note that conditions in the Basic LP do not ensure that two constraints having same scope have same values for the relaxed indicators $\lambda$ over different local assignments. This constraint can be imposed in the following manner.\\
For $C_{i}=(R_i, S_i)$ and $C_j=(R_j, S_j)$, if we have $S_i = S_j$, then
\begin{equation}\label{extracons}
\lambda_i[L] =   \lambda_j[L] ~ \ \forall L \in \mathcal L_i = \mathcal L_j
\end{equation}
However we ignore this condition owing to the difficulty that pops up in using them in a meaningful way while rounding up the fractional solution (to get a full assignment for the original CSP) that we get from the LP relaxation.  
\end{remark}

\subsection{LP as an Expectation Maximization}
Note that between constraints \ref{eq:canonLPmuNonNeg} and \ref{eq:canonLPmuSum}, a solution $\mu^*_v[\ell]$ to the above LP defines a probability distribution (over $\ell \in D$) of assignments for $v$. That is, if we wanted to randomly generate an assignment $\hat{\ell}_v$ for variable $v$, then we could simply state $\hat{\ell}_v = \ell$ with probability $\mu^*_v[\ell]$. Given this interpretation, we write
\begin{equation}
\mu_v[\ell] = \mathbb{P}\left( \hat{\ell}_v = \ell \right) \quad \text{ where } \quad \hat{\ell}_v \sim \mu_v 
\end{equation}

A similar interpretation holds for $\lambda$; between constraints \ref{eq:canonLPlambdaNonNeg} and \ref{eq:canonLPlambdaSum}, a solution $\lambda^*_i[L]$ defines a probability distribution (over $L \in \mathcal{L}_i$) of possible local assignments for constraint $C_i$. That is, if we wanted to randomly select a local  assignment $\hat{L}_i$ for constraint $C_i$, we could state $\hat{L}_i = L$ with probability $\lambda^*_i[L]$. Given this interpretation, we write
\begin{equation}
\lambda_i[L] = \mathbb{P}\left( \hat{L}_i = L \right) \quad \text{ where } \quad \hat{L}_i \sim \lambda_i 
\end{equation}
  
These distributions are tied together by constraint \ref{eq:canonLPConsistency}. In the probabilistic terms established above, constraint \ref{eq:canonLPConsistency} reads
\begin{equation}
\mathbb{P}\left( \hat{\ell}_v = \ell \right) = \sum_{\substack{L \in \mathcal{L}_i \\ L(v) = \ell}} \mathbb{P}\left( \hat{L}_i = L \right) \qquad \forall v \in V, \ell \in D, i : C_i \in C
\end{equation}
Since the events  $\{\hat{L}_i = L_1 \}$ and $\{\hat{L}_i = L_2 \}$ are mutually exclusive, the right hand side can be rewritten to give the following. This simply states that the probability that a variable $v$ takes on value $\ell$ is the same whether you consider the distribution as being defined from $\mu_v$ or $\lambda_i$, for any $i : C_i \in C$. We refer to this constraint as a \textit{first moment consistency constraint}.
\begin{equation} 
\mathbb{P}\left( \hat{\ell}_v = \ell \right) = \mathbb{P}\left( \bigcup_{L : L(v) =  \ell} \left\{\hat{L}_i = L\right\} \right) \qquad \forall v \in V, \ell \in D, i : C_i \in C
\end{equation}

As we remarked in Section \ref{sec:introToCSP} and Equation \ref{expfirst}, we can write the objective function of a CSP in the probabilistic terms $\max_{F} \mathbb{E}_{\tilde{C} \sim W}\left[R_F(\tilde{C})\right] $. With slight modifications (to reflect the fact that our decision variables are now \textit{local} rather than global assignments), we can make a similar statement. 

We define a new random variable $R(\tilde{C}, L(\tilde{C}))$, where $\tilde{C}$ is as defined before i.e., 
$$\mathbb{P} (\tilde{C} = C_i ) = w_i$$
and 
$$\mathbb{P}(L(\tilde{C}) = L |\tilde{C}=C_i ) =  \lambda_i[L]\  \forall L \in \mathcal{L}_i,$$ 
then we can express our objective in \ref{glbobj} as 
\begin{equation}
\max_\lambda \mathbb{E}_{\tilde{C}, L(\tilde{C})} R(\tilde{C}, L(\tilde{C}))
\end{equation}
Using the formula for conditional expectation, we can write this as 
\begin{equation}
\max_\lambda \mathbb{E}_{\tilde{C}} \left[ \mathbb{E} \left(R(\tilde{C}, L(\tilde{C}))\displaystyle|\tilde{C}\right). \right] \label{eq:tower}
\end{equation}
To make the expression more legible we make some abuse of notation and write $C_i \sim W$ in place of $\tilde{C} \sim W$ and write $L \sim \lambda_i$ to denote that $L$ takes values in $\mathcal L_i$ according to $\lambda_i$ given $\tilde{C} = C_i$. And thus our objective is reduced to
\[
	OPT_{LP} = \max_{\lambda_{\mathcal{C}}} \underset{C_i \sim W}{\Ex} \left[ \underset{L \sim \lambda_{i}}{\Ex} \left[ R_i( L (S_i )) |C_i \right] \right]
\]
where $R_i( L (S_i ))$ is an indicator whether assignment $L$ for scope $S_i$ satisfies the relation given by $R_i$.
Having obtained a valid LP relaxation for any CSP problem, the question is now how to use this relaxation to generate good feasible solutions to the CSP. 
A key technique to generate CSP solutions is rounding the LP solution.


\subsection{A Rounding Scheme for the Canonical LP Relaxation}
Various techniques have been developed to extract good CSP solutions from Basic LP.
A common technique to generate CSP solutions with a performance guarantee is randomized rounding.
This technique uses the information available from the LP, typically by using the optimal solution as a probability distribution, to randomly round the each variable to a value in its domain $D$.
To showcase a potential rounding technique, let us consider the problem of Max $k$-SAT. 

\subsubsection{Randomized Rounding scheme for LP relaxation of Max k-SAT}\label{sec:lpRoundingSat}
This rounding scheme was introduced by Goemans and Williamson \cite{GoeWil94}. They combine this randomized rounding scheme together with the seminal approximation algorithm for Max-SAT by Johnson \cite{Joh73} to obtain a $\frac{3}{4}$ bound.
 
Consider the full assignment $F$ as a vector of random variables. For each variable $v \in V$, let the random variable be given by:
\[
	F(v) = \begin{cases}
	1 & \text{with probability } \mu^*_v[1]\\
	0 & \text{with probability } \mu^*_v[0]
	\end{cases}
\]
where $\mu^*_v[\ell]$ is the value of variable $\mu_v[\ell]$ in an optimal solution to the LP.
Since the sum over $\ell \in D = \{0,1\}$ of  $\mu_v[\ell]$ is $1$, and since $\mu_v[\ell]$ is non-negative, the above is a valid definition for a random variable. 
Furthermore, note that $F(1), \dots, F(n)$ is a feasible solution to the Max k-SAT problem.

The expected objective value of this randomized assignment is given by taking expectation in \ref{expfirst} and we get
\[
	\mathbb{E}_F  \left[\text{Val}_{\mathcal{C}}[F] \right]= \mathbb{E}_F \mathbb{E}_{\tilde{C}}\left[R_F(\tilde{C})\right]
\]
Since we can swap the order of expectation, this is equivalent to
\[
	 \mathbb{E}_{\tilde{C}} \mathbb{E}_F \left[R_F(\tilde{C})\right]
\]
and under our new notation we replace $\tilde{C}$ with $C_i$ to get 
\[
	 \mathbb{E}_{C_i \sim W} \mathbb{E}_F \left[R_F(C_i)\right].
\]
Now note that, since $R_F(C_i) = \mathbbm{1} [\mbox{$F$ satisfies $C_i$]}$ we get $\mathbb{E}_F \left[R_F(C_i)\right] = \mathbb{P}_F [\mbox{ $F$ satisfies $C_i$}]$ and our objective function becomes
\[
\mathbb{E}_F  \left[\text{Val}_{\mathcal{C}}[F] \right] = \underset{C_i \sim W}{\mathbb{E}}\left[ \underset{F}{\mathbb{P}}[F \text{ satisfies } C_i] \right]
\]
This swap allows us to consider each $C_i \in \mathcal{C}$ separately.
Therefore, a single constraint $C_i \in \mathcal{C}$ can be considered.

Observe that since the constraints are in disjunctive form, each literal needs to evaluate to false for the constraint to be false. 
In other words, there is a unique falsifying assignment $b_{C_i}$ that makes constraint $C_i$ false. 
For example, if $C_i = x_1 \vee x_2 \vee \bar{x}_3$, then the unique falsifying assignment is given by $x_1 = 0$, $x_2 = 0$, and $x_3 = 1$. 
Considering the definition of $F(v)$ and the independence of $F(i)$ and $F(j)$ for $i \neq j$, it then follows that 
\begin{equation}
		\underset{F}{\mathbb{P}}[ F \text{ satisfies } C_i] = 1 - \prod_{v \in S_i} \mu_v[b_{C_i}(v)] \label{eq:objectiveRounding}
\end{equation}
where $b_{C_i}(v)$ is the value of variable $v \in S_i$ in the falsifying assignment. 
In particular, the probability of not satisfying the constraint in the above example is $ 1- \mu_1^*[0] \mu_2^*[0] \mu_3^*[1]$.

Having identified the objective value of the randomized solution, the question is whether any guarantees can be made on the quality of the solution compared to the LP. 
This can be done, by relating $F$ to the distribution over local assignments.
\begin{align}
	p_{C_i} &:=  \underset{L \sim \lambda_{i}}\Ex \left[ R_i( L (S_i )) |C_i \right]\nonumber\\
			&= \underset{L \sim \lambda_{i}^*}{\mathbb{P}} [L \text{ satisfies } C_i] \nonumber\\
			&=  \underset{L \sim \lambda_{i}^*}{\mathbb{P}}\left[ \bigcup_{v \in S} \left\{ L(v) \neq b_{C_i}(v) \right\} \right] \nonumber\\
			&\le \sum_{v \in  S } \underset{L \sim \lambda_{i}^*}{\mathbb{P}} \left[  \left\{ L(v) \neq b_{C_i}(v) \right\} \right] \nonumber\\
			&= \sum_{v \in S} \left( 1 - \mu_v^*[b_{C_i}(v)] \right) \label{eq:objectiveLP}
\end{align}
This last step follows from the first-order consistency constraints:
\[
	\underset{L \sim \lambda_{i} }{\mathbb{P}}[ L(v) = \ell] = \underset{L \sim \mu_v }{\mathbb{P}}[ L = \ell]
\]

It remains to relate the LP objective value to the expected objective of the rounding procedure. 
Note that the former (see \eqref{eq:objectiveLP}) is an arithmetic mean (AM) whereas the latter (see \eqref{eq:objectiveRounding}) is an geometric mean (GM). 
This suggests the use of the inequality of arithmetic and geometric means.
\begin{align*}
		\underset{F}{\mathbb{P}}[ F \text{ satisfies } C_i] &= 1 - \prod_{v \in S_i} \mu_v[b_{C_i}(v)]\\
		&= 1 - \underset{v \in S_i}{GM}(\mu_{v}^*[b_c(v)])^{|S_i|}\\
		&\ge 1 - \underset{v \in S_i}{AM}(\mu_{v}^*[b_c(v)])^{|S_i|}\\
		&= 1 - \left( 1  - \underset{v \in S_i}{AM}(1 -\mu_{v}^*[b_c(v)]) \right)^{|S_i|}\\
		&\ge 1 - \left( 1  - \frac{p_{C_i}}{|S_i|}\right)^{|S_i|}\\
\end{align*}
One can show that $$1 - \left( 1  - \frac{p_{C_i}}{|S_i|}\right)^{|S_i|} \geq r \cdot p_{C_i} ~\ \forall p_{C_i} \in [0, 1]$$ and $r \downarrow 1-1/e$ as $|S_i| \rightarrow \infty$.Therefore, the objective value of the rounding scheme
\begin{align*}
	\underset{F}{\Ex} \left[ \text{Val}_{\mathcal{C}}[F]\right] &= \underset{F}{\Ex}\left[ \underset{C_i \sim W}{\mathbb{P}}[ F \text{ satisfies } C_i] \right]\\
	&=  \underset{C_i \sim W}{\mathbb{E}}\left[ \underset{F}{\mathbb{P}}[ F \text{ satisfies } C_i] \right]\\
	&\ge \left( 1 - \frac{1}{e} \right) \underset{C_i \sim W}{\Ex} [ p_{C_i} ]\\
	&= \left( 1 - \frac{1}{e} \right) \text{OPT}_{LP}(\mathcal{C})\\
	&\ge \left( 1 - \frac{1}{e} \right) \text{OPT}(\mathcal{C})\\
\end{align*}

Hence, we conclude with the following theorem:
\begin{thm}
	Randomized rounding is a $\left( ( 1 - 1/e)\beta, \beta \right)$-approximation for Max-SAT for any $\beta$.
\end{thm}
\subsection{CSP-Specific LP Relaxations and Rounding Schemes}
Note the following specific relaxations \cite{GoeWil94,Asa97,Yan94}.
\begin{itemize}
\item this thing
\item this thing as well
\item also this thing
\end{itemize}

