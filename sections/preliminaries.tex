\section{Preliminaries}
This section provides a brief overview of the material that is necessary to understand this report. There are three main requirements
\begin{itemize}
\item Some familiarity with the fundamentals of algorithmic complexity.
\item Exposure to semidefinite and linear programming.
\item Proficiency in linear algebra and probability.
\end{itemize}

With regards to algorithmic complexity, we assume the reader is comfortable with big-O notation for algorithm runtime, as well as NP-Completeness and NP-Hardness. 
Although it is expected that the reader be comfortable with these topics, key terms in complexity theory are defined below for convenience.

Linear algebra is not required to understand the notion of a Constraint Satisfaction Problem, but will be necessary to understand the semidefinite programming approaches that are used to approximate CSP's. 
Probability also plays a key role in these semidefinite programming approaches.

A reader with some knowledge of ``approximation algorithms" is likely to better appreciate the material in this report, but strictly speaking, the prior knowledge is not necessary. 
We provide a very brief introduction to approximation algorithms below. 
Those experienced in the field should pay special attention to Definition \ref{def:twoParamApproxAlg}, which has seen only limited use.

\subsection{NP-Hardness and NP-Completeness}
\begin{ILdefinition}
A \textbf{decision problem} is a yes-no question.
Given some input data, and a set of rules, does the input data satisfy the rules?
\end{ILdefinition}
It is implied that sufficient data is provided to definitively answer ``yes" or ``no", even if determining the answer might take a prohibitively long amount of time.
\begin{ILdefinition}
\textbf{NP} is a collection of decision problems. A decision problem is in \textbf{NP} if for every ``yes" answer, there is an efficient procedure to verify that the answer is in fact ``yes." 
There are no requirements for ``no" answers.
\end{ILdefinition}
\begin{ILdefinition}
A decision problem $\mathcal{P}$ is said to be \textbf{NP-Complete} if (1)
$\mathcal{P} \in $ NP, and (2) any other problem $\mathcal{Q} \in$ NP can be stated ``succinctly" in terms of $\mathcal{P}$ .
\end{ILdefinition}
By ``succinctly", we mean that the transformation from $\mathcal{Q}$ to the equivalent problem in the terms of $\mathcal{P}$ can be done in both polynomial time and space.\footnote{This process of carrying out this transformation is usually called a ``reduction."}
\begin{ILdefinition}
A problem $\mathcal{P}$ (which may or may not be a decision problem) is said to be \textbf{NP-Hard} if every problem in NP can be stated succinctly in terms of $\mathcal{P}$.
\end{ILdefinition}

\subsection{Approximation Algorithms for Optimization Problems}\label{subsec:intractProbCope}

Although ``problems" are said to be intractable, we actually only solve problem \textit{instances}. 
In a variety of circumstances, it is reasonable to solve an instance of an intractable problem with an exact but exponential algorithm.

For example, there is billions of dollars of capital involved in coordinating the movements of even a handful of trans-oceanic shipping vessels. 
Although not necessarily the case, it is very likely that such a logistics problem would be NP-Hard.
Nevertheless, if there are a sufficiently small number of decisions to be made in this planning process, it could make perfect sense to solve the planning problem with an exact algorithm.

The situation changes slightly when deadlines are involved, since getting \textit{some} solution by a deadline is often more important than getting the \textit{best} solution after that time. 
The presence of imminent deadlines does not completely rule out the use of exact algorithms; high powered computers with sophisticated (but still exponential) algorithms are often used under these circumstances. 
Logistics for airlines is one prominent example.

But when an intractable problem has thousands of variables, exact methods are typically worthless. 
For those facing intractable problems of this scale, algorithms which provide solutions within a reasonable amount of time are of paramount importance.  
When the algorithm has some guarantee on the quality of its solution, it is referred to as an ``approximation algorithm."
We give a definition below in the case where the objective is to maximize some function. 

\begin{definition}
\textbf{($\alpha$)-Approximation Algorithm} \\
Let $\Omega$ denote the set of all possible instances of a given maximization problem. 
Let $A$ denote an efficient algorithm which returns a feasible but potentially sub-optimal solution for any $I \in \Omega$. 
Denote the value of the solution returned by $A$ on $I$ as $A(I)$, and denote the value of the optimal solution for $I$ by $OPT(I)$. We call $A$ an $\alpha$-approximation if
\begin{equation*}
\frac{OPT(I)}{A(I)} \leq \alpha ~ \forall I \in \Omega
\end{equation*}
\label{def:commonApproxAlg}
\end{definition}

Definition \ref{def:commonApproxAlg} is the most common definition of an approximation algorithm, and is suitable for many applications. It can be useful, however, to describe how performance guarantees relate to the optimal objective value. 
\begin{definition}
\textbf{$(\alpha,\beta)$-Approximation Algorithm } \\
Let $\Omega$, $I$, and $A$ be as before. 
Define $\Omega_\beta$ as the set of all problem instances with $OPT(I) \geq \beta \forall I \in \Omega_\beta$.
We call $A$ an $(\alpha,\beta)$ approximation if 
\begin{equation*}
A(I) \leq \alpha ~ \forall I \in \Omega_{\beta}
\end{equation*}
contrary to $(\alpha)-$approximation algorihtms, ``$\alpha$" here is \textit{not} defined as a ratio.
\label{def:twoParamApproxAlg}
\end{definition}
\newpage