\documentclass[letterpaper, 12pt]{article}

\usepackage[margin=2.5cm]{geometry}
\usepackage{amsmath,amsthm,amssymb}
\usepackage[]{mathtools}
\usepackage[]{bbm}

\numberwithin{equation}{section}

% --to donotes
\usepackage{xargs}                      % Use more than one optional parameter in a new commands
\usepackage[dvipsnames]{xcolor}
\usepackage[colorinlistoftodos,prependcaption,textsize=tiny]{todonotes}
\newcommandx{\unsure}[2][1=]{\todo[linecolor=red,backgroundcolor=red!25,bordercolor=red,#1]{#2}}
\newcommandx{\change}[2][1=]{\todo[linecolor=blue,backgroundcolor=blue!25,bordercolor=blue,#1]{#2}}
\newcommandx{\info}[2][1=]{\todo[linecolor=OliveGreen,backgroundcolor=OliveGreen!25,bordercolor=OliveGreen,#1]{#2}}
\newcommandx{\maybeinclude}[2][1=]{\todo[linecolor=Orange,backgroundcolor=Orange!25,bordercolor=Orange,#1]{#2}}
\newcommandx{\improvement}[2][1=]{\todo[linecolor=Plum,backgroundcolor=Plum!25,bordercolor=Plum,#1]{#2}}

\usepackage[]{thmtools}
\usepackage[dvipsnames]{xcolor}
\declaretheoremstyle[
	bodyfont=\normalfont, 
	spaceabove=0.5cm]{defFormat}
\declaretheoremstyle[
	postheadspace=1cm,
	bodyfont=\normalfont, 
	spaceabove=0.5cm]{inLineDefFormat}
\declaretheoremstyle[
	spaceabove=0.5cm, 
	spacebelow=0.5cm,
	postheadspace=1cm]{namedTheorem}
\declaretheorem[
	numberwithin=section, 
	style=defFormat, 
	shaded]{definition}
\declaretheorem[
	style=inLineDefFormat, 
	sibling=definition,
	shaded,
	name=Definition]{ILdefinition}
\declaretheorem[
	numbered=no, 
	style=namedTheorem, 
	shaded, 
	name=The Unique Games Conjecture]{ugc}
\newtheorem{thm}{Theorem}
\declaretheorem[numberwithin=section, style=defFormat, shaded]{Definition}

% Shortcuts:

%Theorems
\newtheorem{theorem}{Theorem}
\newtheorem{lemma}{Lemma}
\newtheorem{prop}{Proposition}
\newtheorem{remark}{Remark}
\newtheorem{example}{Example}
%
%Environments
\newcommand{\be}[1]{\begin{eqnarray}#1\end{eqnarray}}
\newcommand{\al}[1]{\begin{align}#1\end{align}}
\newcommand{\iit}[1]{\begin{itemize}#1\end{itemize}}
\newcommand\bmat[1]{\begin{bmatrix}#1\end{bmatrix}}

%Genuine Shortcuts
\newcommand{\lrr}{\Longleftrightarrow}
\newcommand{\rr}{\Rightarrow}
\newcommand{\fa}{\ \forall \ }

\newcommand{\Rn}[1]{\mathbb{R}^{#1}}
\newcommand{\Pb}{\mathbb{P}}
\newcommand{\Ex}{\mathbb{E}}
\newcommand{\R}{\mathbb{R}}
\newcommand{\N}{\mathcal{N}}
\newcommand{\C}{\mathcal{C}}
\newcommand{\Sm}{\mathcal{S}}
\newcommand{\Ra}{\mathcal{R}}
\newcommand{\D}{\Delta}
\newcommand{\G}{\Gamma}
\newcommand{\La}{\Lambda}
\newcommand{\la}{\lambda}
\newcommand{\Si}{\Sigma}

\newcommand{\simax}{{\sigma_{\mathrm{max}}}}
\newcommand{\simin}{{\sigma_{\mathrm{min}}}}
\newcommand{\lmax}{{\lambda_{\mathrm{max}}}}
\newcommand{\lmin}{{\lambda_{\mathrm{min}}}}

\newcommand{\prob}{{\mbox{\rm Prob}}}
\newcommand{\var}{{\mbox{\rm var}}}
\newcommand{\sint}{{\mbox{\rm int}\,}} %set interior
\newcommand{\relint}{{\mbox{\rm relint}\,}} %set interior
\newcommand{\ns}{{\mbox{\tt ns}}} 

\newcommand{\lrb}[1]{\left(#1\right)}
\newcommand{\lrbb}[1]{\left[#1\right]}
\newcommand{\inv}{^{-1}}
\newcommand{\trans}{^T}

\newcommand{\os}[1]{\overset{#1}}
\newcommand{\us}[1]{\underset{#1}}
\newcommand{\sgn}[1]{\mbox{sgn}\lrb{#1}}
\newcommand{\tr}[1]{\mbox{Tr}\lrb{#1}}


\DeclareMathOperator{\grad}{\nabla}  % gradient

\newcommand{\Real}[1]{ { {\mathbb R}^{#1} } }

\renewcommand{\vec}[1]{\mathbf{#1}}
\newcommand{\norm}[1]{\lVert #1 \rVert}
\DeclareMathOperator{\Tr}{Tr}

\parindent 0pt
\parskip 8pt

\begin{document}
	\begin{center}
		{
		\bf \huge SDP Relaxations for Combinatorial Optimization
		}
	\end{center}
	% Reader introduction
	\begin{abstract}

The aims of the report are threefold.  (1) Introduce the reader to the Constraint Satisfaction Problems (CSP) framework. (2) Equip the reader with ``tractable tools" from convex optimization (and randomized algorithms) to find a ``good" solution. This is done in a wholesome fashion since we don't leave for the reader to implement the schemes but in fact demonstrate them in our numerical section. (3) Introduce some of the optimal approximation schemes in literature and comment on their computational aspects.

With regards to introducing the reader to Constraint Satisfaction Problems, we note that different communities mean different things by ``Constraint Satisfaction Problem". Our definition is motivated by recent developments in approximation algorithms for certain combinatorial problems. Once we have clearly specified the CSP framework, we provide a brief overview the Unique Games Conjecture, and how it connects to approximating CSP's.

Once the CSP framework and its relation to the Unique Games Conjecture is established, we present a tractable linear programming relaxation with a valuable probabilistic interpretation. The probablistic interpretation of the LP is crucial in understanding subsequent SDP formulation, \textit{but we present the SDP first without this interpretation} (we believe that doing so makes it easier to understand how to specify the SDP to a software package like CVX). The majority of this report is dedicated to SDP relaxations for CSP's.

\end{abstract}
	\newpage	
	% prelim
	\section{Preliminaries}
The study of Constraint Satisfaction Problems is of significant interest precisely because many practical problems are intractable.  

It is assumed that the reader has some familiarity fundamentals of algorithmic complexity, including big-Oh notation for algorithm runtime, as well as NP-hardness and NP-completeness. 
For the reader's convenience, we define some key terms in complexity theory below.

A reader with some knowledge of ``approximation algorithms" is likely to better appreciate the material in this report, but strictly speaking, the prior knowledge is not necessary. 
We provide a very brief introduction to approximation algorithms below. 
Those experienced in the field should pay special attention to Definition \ref{def:twoParamApproxAlg}, which has seen only limited use.

The focus of this report will be the use of Semidefinite Programming for CSP's. As such, the reader is expected to be proficient in linear algebra and to have some exposure to mathematical programming (linear programming at minimum).

\subsection{NP-Hardness and NP-Completeness}
\begin{ILdefinition}
A \textbf{decision problem} is an algorithmic question.
Given some input data, and a set of rules, does the input data satisfy the rules?
\end{ILdefinition}
It is implied that sufficient data is provided to definitively answer ``yes" or ``no", even if determining the answer might take a prohibitively long amount of time.
\begin{ILdefinition}
\textbf{NP} is a collection of decision problems. A decision problem is in \textbf{NP} if for every ``yes" answer, there is an efficient procedure to verify that the answer is in fact ``yes." 
There are no requirements for ``no" answers.
\end{ILdefinition}
\begin{ILdefinition}
A decision problem $\mathcal{P}$ is said to be \textbf{NP-Complete} if (1)
$\mathcal{P} \in $ NP, and (2) any other problem $\mathcal{Q} \in$ NP can be stated ``succinctly" in terms of $\mathcal{P}$ .
\end{ILdefinition}
By ``succinctly", we mean that the transformation from $\mathcal{Q}$ to the equivalent problem in the terms of $\mathcal{P}$ can be done in both polynomial time and space.\footnote{This process of carrying out this transformation is usually called a ``reduction."}
\begin{ILdefinition}
A problem $\mathcal{P}$ (which may or may not be a decision problem) is said to be \textbf{NP-Hard} if every problem in NP can be stated succinctly in terms of $\mathcal{P}$.
\end{ILdefinition}

\subsection{Approximation Algorithms for Optimization Problems}\label{subsec:intractProbCope}

Although ``problems" are said to be intractable, we actually only solve problem \textit{instances}. 
In a variety of circumstances, it is reasonable to solve an instance of an intractable problem with an exact but exponential algorithm.

For example, there is billions of dollars of capital involved in coordinating the movements of even a handful of trans-oceanic shipping vessels. 
Although not necessarily the case, it is very likely that such a logistics problem would be NP-Hard.
Nevertheless, if there are a sufficiently small number of decisions to be made in this planning process, it could make perfect sense to solve the planning problem with an exact algorithm.

The situation changes slightly when deadlines are involved, since getting \textit{some} solution by a deadline is often more important than getting the \textit{best} solution after that time. 
The presence of imminent deadlines does not completely rule out the use of exact algorithms; high powered computers with sophisticated (but still exponential) algorithms are often used under these circumstances. 
Logistics for airlines is one prominent example.

But when an intractable problem has thousands of variables, exact methods are typically worthless. 
For those facing intractable problems of this scale, algorithms which provide solutions within a reasonable amount of time are of paramount importance.  
When the algorithm \textit{does} have a performance guarantee, it is referred to as an ``approximation algorithm."
We give a definition below in the case where the objective is to maximize some function. 

\begin{definition}
\textbf{($\alpha$)-Approximation Algorithm} \\
Let $\Omega$ denote the set of all possible instances of a given maximization problem. 
Let $A$ denote an efficient algorithm which returns a feasible but potentially sub-optimal solution for any $I \in \Omega$. 
Denote the value of the solution returned by $A$ on $I$ as $A(I)$, and denote the value of the optimal solution for $I$ by $OPT(I)$. We call $A$ an $\alpha$-approximation if
\begin{equation*}
\frac{OPT(I)}{A(I)} \leq \alpha ~ \forall I \in \Omega
\end{equation*}
\label{def:commonApproxAlg}
\end{definition}

Definition \ref{def:commonApproxAlg} is the most common definition of an approximation algorithm, and is suitable for many applications. It can be useful, however, to describe how performance guarantees relate to the optimal objective value. 
\begin{definition}
\textbf{$(\alpha,\beta)$-Approximation Algorithm } \\
Let $\Omega$, $I$, and $A$ be as before. 
Define $\Omega_\beta$ as the set of all problem instances with $OPT(I) \geq \beta \forall I \in \Omega_\beta$.
We call $A$ an $(\alpha,\beta)$ approximation if 
\begin{equation*}
\frac{OPT(I)}{A(I)} \leq \alpha ~ \forall I \in \Omega_{\beta}
\end{equation*}
\label{def:twoParamApproxAlg}
\end{definition}
\newpage
	\newpage
	% CSP introduction
	%\documentclass[12pt]{article}
%\usepackage{fullpage}
%\usepackage[]{mathtools}
%\usepackage{amsthm}
%\DeclareMathOperator{\Tr}{Tr}
%\usepackage[]{thmtools}
%\usepackage[dvipsnames]{xcolor}
%\declaretheorem{theorem}
%\declaretheoremstyle[
%	bodyfont=\normalfont, 
%	spaceabove=0.5cm]{defFormat}
%\declaretheoremstyle[
%	postheadspace=1cm,
%	bodyfont=\normalfont, 
%	spaceabove=0.5cm]{inLineDefFormat}
%\declaretheoremstyle[
%	spaceabove=0.5cm, 
%	spacebelow=0.5cm,
%	postheadspace=1cm]{namedTheorem}
%\declaretheorem[
%	numberwithin=section, 
%	style=defFormat, 
%	shaded]{definition}
%\declaretheorem[
%	style=inLineDefFormat, 
%	sibling=definition,
%	shaded,
%	name=Definition]{ILdefinition}
%\declaretheorem[
%	numbered=no, 
%	style=namedTheorem, 
%	shaded, 
%	name=The Unique Games Conjecture]{ugc}
%\declaretheorem[numbered=no, style=remark]{remark}
%\usepackage[]{bbm}
%\usepackage[]{amssymb}
%\begin{document}

%\parindent 0pt
%\parskip 8pt

\section{Introduction to Constraint Satisfaction Problems}

\subsection{Getting Oriented with CSP's}

There is a great deal of research going into the approximabilty of constraint satisfaction problems, and some claims regarding CSP's can seem quite sensational. 
The list of points below gives some facts relating to CSP's to help orient the reader.

\begin{itemize}
\item At their core, CSP's are \textit{optimization} problems. But as we will see, CSP's can be used to approach \textit{decision} problems.
\item The objective of a CSP is to satisfy as many constraints as possible; there is only one constraint that is ``safe" from violation: all variables must be in some simple, discrete domain.
\item Most CSP's are NP-Hard to solve optimally. Because of this, discussion of CSP's centers on developing approximate solutions to these problems.
\item Semidefinite Programming is the primary technique for CSP approximation.
\item The ability to generate a near optimal-solution for a CSP is intimately related to an open problem in computer science known as the ``Unique Games Conjecture."
\end{itemize}

\subsection{The CSP Framework and CSP Instances}
Define the \textit{arity} of an indicator function $R$ as the number of arguments it takes, and denote the arity by $\text{ar}(R)$.

\begin{definition}
\textbf{The CSP Framework} \\
Let $D$ be a finite domain of fixed cardinality $q$. 
Let $R$ denote an indicator function over $D$ with arity $r \leq k$ (i.e. $R:D^r \to \{0,1\}$). 
Let $\Gamma$ be a possibly exhaustive set of such functions. 
$D$ and $\Gamma$ define a \textit{class} of problems which we denote CSP($\Gamma$).
\label{def:CSPframework}
\end{definition}

We usually write $D = \{0,1,\ldots,q-1\}$, although the elements of $D$ can serve as \textit{labels}, without any of the algebreic structure implied by the use of integers.

\textbf{Examples}
\begin{itemize}
\item Max-Cut : For a given graph, label each vertex either ``$0$" or ``$1$" to maximize the number of edges connecting verticies with different labels.   
$D = \{0,1\}$, and all constraints are of the form $\{v_i \neq v_j\}$. We could write $\Gamma = \{\neq\}$ as the ``not equal" operator. 
\item $q-$Coloring : Find an a as-consistent-as-possible coloring for a graph using at most $q$ colors. 
Our domain is $D = \{0,1,\ldots,q-1\}$, and $\Gamma$ is again $\{\neq\}$. 
Note that Max-Cut is equivalent to $q-$coloring with $q = 2$.
\item Max E3-SAT : Here, the domain is $D = \{0,1\}$, and our constraint types (elements of $\Gamma$) are of the form $(x_j \vee x_j \vee x_k)$, $(x_i\vee \bar{x}_j \vee x_k)$, $(\bar{x}_i \vee \bar{x}_j \vee x_k)$, $\ldots$ - all disjunctions on \textit{`E'xactly} $3$ literals. 
\item Max $k-$SAT : $D = \{0,1\}$, $\Gamma$ is all disjunctions of \textit{up to} $k$ literals.
\item $3-$CSP : $D = \{0,1\}$, $\Gamma$ is all relations on up to three binary variables.
\end{itemize}

\begin{definition}
\textbf{CSP Instance}\\
An instance $\mathcal{C}$ of CSP($\Gamma$) is characterized by a set of $n$ variables (denoted $V$), as well as $m$ constraints and $m$ positive weights (one for each constraint). 
Every constraint $C_i \in \mathcal{C}$ has the form $C_i = (R_i,S_i)$ where $R_i \in \Gamma$ and $S_i$ (said to be the \textit{scope} of $C_i$) is a possibly ordered list of \text{ar}($R_i$) variables. 

If $F$ (a mapping $ V\to D$) is a given assignment of variables, then $R_i(F(S_i))$ equals either 1 or 0, in which case constraint $C_i$ is said to be ``satisfied" or ``not satisfied" respectively.
The objective is to maximize the weighted sum of satisfied constraints $\sum_{i = 1}^m w_i R_i(F(S_i))$. 

$\mathcal{C}$ is said to be \textit{satisfiable} if $\forall i, ~ R_i(F(S_i)) = 1$ at optimality.
\label{def:CSPinstance}
\end{definition}
To simplify notation, we will often write $F(C_i)$ in place of $R_i(F(S_i))$.

\subsubsection{Discussion of Definition \ref{def:CSPinstance}}
\textbf{Remarks on the objective function - }Without loss of generality, we may take the weights to sum to 1. 
When we do this, we can interpret the weights as probabilities and $C$ as a random variable ($C = C_i$ with probabilty $w_i$). 
In this notation, we can write any CSP instance as
\begin{equation}
\max_{F: V \to D} \mathbb{E}\left[F(C)\right]
\end{equation}

\textbf{Remarks on the constraints - }In almost all optimization paradigms, ``constraints" are by their very definition \textit{inviolable}. 
Definition \ref{def:CSPinstance} departs from this convention in that a feasible solution is not required to satisfy any of a CSP's ``constraints." 
Are CSP's then accurately understood as unconstrained optimization? 
Not quite. 
The CSP Framework has one and only one inviolable constraint (and that constraint is precisely what makes them difficult): each of the $n$ variables $x_i, ~ i \in\{1,\ldots,n\}$ belongs to the discrete domain $D$. 

\subsection{Complexity for Solving and Approximating CSP's}
One of the most surprising results in CSP's and approximation algorithms is the lack of a consistent relationship between the difficulty of solving a problem optimally, and the difficulty of approximating the problem.

Take 3-SAT for example. 
It is well known that it is NP-Hard to find a satisfiable assignment to a 3-SAT instance even \textit{given} that it is satisfiable. 
It is equivalent to say that it is NP-Hard to $(1,1)$ approximate ``Max 3-SAT." 
In spite of this, Max 3-SAT has a well known $(\frac{7}{8}\beta,\beta)$ polynomial time approximation algorithm - for all $ \beta \leq 1$! 
From this we might expect that approximating an optimization problem (Max 3-SAT) is easier than solving the corresponding decision problem (3-SAT). 
Alas, this is not always the case. 

For example, the 2-SAT decision problem can be solved in polynomial time, but Max 2-SAT is NP-Hard to approximate by better than 0.955. 
The Max-Cut problem is yet another example of this phenomenon. 
There is a polynomial time algorithm for determining whether or not a Max-Cut instance is satisfyable\footnote{one need only check whether the graph is bipartite}, 
but the best approximation algorithm for Max-Cut in the general case is a $(0.878\beta,\beta)$ approximation. 
It has been proven that it is NP-Hard to approximate Max-Cut by a factor better than 0.942, 
but it may yet be NP-Hard to do better than 0.878! 

Whether existing approximation algorithms can be improved for a huge swath of combinatorial problems (including Max-Cut), depends heavily on the truth of the Unique Games Conjecture. 
We address this next.

\subsection{Hardness of Approximation: The Unique Games Conjecture} 
In 2002, Subhash Khot published a paper entitled \textit{On the Power of Unique 2-Prover 1-Round Games}. 
In his paper, Khot put forward the Unique Games Conjecture- currently of the most important open problems in theoretical computer science.

It can be difficult to get a handle on what the Unique Games Conjecture claims. 
The purpose of this section is to provide the reader with the background necessary to understand how the Unique Games Conjecture relates to CSP approximation. 
This will become critically important in subsequent sections on SDP relaxations of CSP's.

\begin{itemize}
\item The UGC is a statement of the hardness of approximation: 
for some problems, it is NP-Hard to determine whether every solution for a given instance is \textit{extremely poor}, or whether an \textit{almost perfect} solution exists for the instance.
\item If true, the UGC would imply that many already existing approximation algorithms cannot be improved upon. 
This includes approximation algorithms which are not for CSP's and do not use semidefinite programming.
\end{itemize}

\subsubsection{What is a ``Unique Game"?}

A ``unique game" is a game in the game-theory sense that relates to Probabilistically Checkable Proofs (PCP's). 
It is a slightly less general version of ``2-Prover 1-Round games." 
Below we list the three conditions of 2-Prover 1-Round games, and then the additional condition that makes such a game ``unique."
\begin{itemize}
\item The game pits two \textit{provers} against a \textit{verifier}.
\item The verifier asks one question of each prover.\footnote{Questions are drawn from appropriate probability distributions.}
\item Given the answers returned by the two provers, the verifier returns ``True" or ``False."
\item[\textbf{*}] The answer of one prover \textit{uniquely} determines the answer of the other prover.
\end{itemize}

\subsubsection{What does UGC have to do with CSP's or combinatorial optimization?}

Khot posed the ``game" in the UGC in three equivalent ways
\begin{itemize}
\item The 2-Prover 1-Round game with uniqueness (specified above).
\item A CSP where the constraints to be satisfied are a system of linear equations in two variables, modulo 2.
\item A new graph-theoretic CSP called ``Label Cover."\footnote{Khot's game chose a particular type of Label Cover problem where the graph is bipartite, but subsequent discussions of Unique Games do not emphasize this characteristic. }
\end{itemize}

If it is claimed that an algorithm is for ``Unique Games," the reader would do well to clarify which formation the author's work with. Because graph theory has a larger research community than those working with Probabilistically Checkable Proofs, several researchers (including Khot himself) primarily use the Label Cover formulation of Unique Games in discourse. 

The interested reader is strongly recommended to read Khot's original paper for a precise definitions for Proababilistically Checkable Proofs, the Two-Prover One-Round ``unique" game, and the Label Cover problem. But suffice it to say

\begin{ugc}
For arbitrarily small constants $\epsilon$, $\delta > 0$, there exists a constant $k = k(\epsilon,\delta)$ such that it is NP-Hard to determine whether a unique Label Cover instance with label sets of size $k$ at satisfies \textit{at least} $1-\epsilon$ or \textit{at most} $\delta$ constraints.
\end{ugc}




%\end{document}

	\newpage
	% lp relaxation
	\section{Linear Programming for CSP Approximation}\label{sec:lpRelax}
The goal of this section is to present a ``canonical" LP relaxation and rounding scheme for Constraint Satisfaction Problems. Both the relaxation and rounding scheme are valid for arbitrary CSP's, although performance guarantees are only proven in limited cases.  A non-exhaustive list of CSP-specific LP relaxations is included at the end of this section.

\subsection{An Integer Program : Towards the Canonical LP [CITE]}\label{subsec:ip}
Let $\mathcal{C} = (V,C,W)$ be a CSP over a domain $D$ of size $q$.

Earlier, we defined an \textit{assignment of variables} as a function $F : V \to D$. Critically, the domain of $F$ is the entire set $V$. When this is the case, we could call $F$ a \textit{full assignment}. It is reasonable (and as we will see, helpful!) to consider $F$ as being built from many \textit{local assignments} $L : S \to D$ where $S \subset V$. For a constraint $C_i = (R_i,S_i) \in \mathcal{C}$, we will be interested in the local assignment $L : S_i \to D$. Where before we could write $R_i(F(S_i))$ as the value of the constraint under an assignment, we write $R_i(L)$ when it is given that $L$ is a local assignment for constraint $C_i$.

Now consider an \textit{Integer}-Linear Program over the following variables:
\begin{itemize}
\item $\mu_v[\ell] \in \{0,1\}$ is an indicator that variable $v \in V$ takes value $\ell \in D$
\item $\lambda_i[L] \in  \{0,1\}$ is an indicator that \textit{local assignment} $L$ is used for constraint $C_i$
\end{itemize}
From these definitions, it is clear that for fixed $v$, we need one and only one $\mu_v[\ell]$ to be equal to 1. Encode this constraint as 
\begin{equation}\label{musum}
\sum_{\ell \in D} \mu_v[\ell] = 1.
\end{equation}

Now we define $\mathcal{L}_i$ as the set of all possible local assignments for the variables in constraint $C_i$'s scope. We note that the size of $\mathcal{L}_i$ is exponential in $t = ar(R_i)$ (in fact, it's exactly $|\mathcal{L}_i| = q^t$). This is one of the key reasons why maximum arity is a \textit{fixed parameter} for all $\mathcal{C} \in \text{CSP}(\Gamma)$.

Since any two local assignments $L_1$, $L_2$ in $ \mathcal{L}_i$ are mutually exclusive, we likewise need one and only one of $\lambda_i[L]$ equal to 1 for fixed $i$.
\begin{equation}\label{lambsum}
\sum_{L \in \mathcal{L}_i} \lambda_i[L] = 1 
\end{equation}\label{mulambcons}
To be consistent across $\lambda$ and $\mu$, we need one more constraint.
\begin{equation}
\mu_v[\ell] = \sum_{\substack{ L \in \mathcal{L}_i \\ L(v) = \ell }} \lambda_i[L] 
\end{equation}
Now we need to come up with an objective that mimics the one defined in Equation \ref{expfirst}. We re-write the expression for objective for clarity: 
$$\max_F \sum_{i:C_i \in C} w_i R_i(F(S_i))$$
As discussed in the beginning of this section it is easy to see 
\begin{equation}\label{replacewithlocal}
R_i(F(S_i)) = \sum_{L\in \mathcal L_i}\lambda_i[L] R_i(L).
\end{equation} 
It follows from the facts that whether or not $R_i$ is satisfied, depends only on the local assignment and RHS of \ref{replacewithlocal} is precisely these terms summed over all possible local assignments with multiplicative factor of indicators for the respective assignments. And thus our objective becomes 
\begin{equation}\label{glbobj}
\max_{\mu, \lambda} \sum_{i : C_i \in C} \sum_{L \in \mathcal{L}_i}   w_i\lambda_i[L] R_i(L).
\end{equation}

We get the corresponding LP simply by relaxing $\mu_v[\ell] \in \{0,1\}$ to $\mu_v[\ell] \in [0,1]$ and $\lambda_i[L] \in \{0,1\}$ to $\lambda_i[L] \in [0,1]$.


\subsection{The Canonical Linear Program}

We define the linear program below, then address its probabilistic interpretation and equivalent representations.
\begin{definition}\textbf{Basic LP} \\
Let $\mathcal{C} = (V,C,W)$ be a CSP over domain $D$. The Basic LP for $\mathcal{C}$ is
\begin{alignat}{2}
\max_{\mu, \lambda} ~&~ \sum_{i : C_i \in C} \sum_{L \in \mathcal{L}_i}   w_i\lambda_i[L] R_i(L) & \\
s.t. ~ & ~ \sum_{\ell \in D} \mu_v[\ell] = 1 & \forall v \in V  \label{eq:canonLPmuSum} \\
     ~ & ~ \sum_{L \in \mathcal{L}_i} \lambda_i[L] = 1  & \forall i : C_i \in C \label{eq:canonLPlambdaSum} \\
     ~ & ~ \sum_{\substack{ L \in \mathcal{L}_i \\ L(v) = \ell }} \lambda_i[L] = \mu_v[\ell]  & \forall v \in V, \ell \in D, i : C_i \in C \label{eq:canonLPConsistency} \\
     ~ & ~ 0 \leq \mu_v[\ell] \leq 1 & v \in V, \ell \in D \label{eq:canonLPmuNonNeg}\\
     ~ & ~ 0 \leq \lambda_i[L] \leq 1  & \forall  i : C_i \in C, L \in \mathcal{L}_i  \label{eq:canonLPlambdaNonNeg} 
\end{alignat}
\end{definition}

\begin{lemma}\label{le:super}
For the Basic LP, constraints \ref{eq:canonLPmuSum} and \ref{eq:canonLPmuNonNeg} are redundant. 
\end{lemma}
\begin{proof}
It is not difficult to see that constraints \ref{eq:canonLPmuSum} and \ref{eq:canonLPmuNonNeg} are redundant in the presence of \ref{eq:canonLPlambdaSum}, \ref{eq:canonLPConsistency} and \ref{eq:canonLPlambdaNonNeg} :
\begin{align*}
\ref{eq:canonLPlambdaNonNeg} \rm{\ and \ } \ref{eq:canonLPlambdaSum} &\Rightarrow 0\leq  \sum_{\substack{ L \in \mathcal{L}_i \\ L(v) = \ell }} \lambda_i[L] \leq  \sum_{L \in \mathcal{L}_i} \lambda_i[L] = 1  \\
{\rm \ Combine\ with \ }\ref{eq:canonLPConsistency}&\Rightarrow  0\leq \mu_v[L] \leq 1
\end{align*}
Thus, \ref{eq:canonLPmuNonNeg} is redundant. 

Also, summing on both sides of \ref{eq:canonLPConsistency} with respect to $\ell \in D$ we get, 
\begin{align*}
{\rm RHS} = \sum_{\ell \in D} \mu_v[\ell] = {\rm LHS} = \sum_{\ell \in D} \sum_{\substack{ L \in \mathcal{L}_i \\ L(v) = \ell }} \lambda_i[L] = \sum_{L \in \mathcal{L}_i} \lambda_i[L] \overset{\ref{eq:canonLPlambdaSum}}=1
\end{align*}
and thus even \ref{eq:canonLPmuSum} is superfluous.
\end{proof}

In other words, the remark above renders $\mu_v[\cdot]$ indicators as superfluous. \emph{Nevertheless, we keep them for the discussion that follows, as they help in gaining some insight, help us come up with an intuitive rounding scheme and also lead us to extending the relaxation in a very natural fashion.}

\begin{thm}
	Basic LP is a relaxation for the problem CSP$(\Gamma)$.
\end{thm}
\begin{proof}
	Given an optimal solution for an instance of the problem CSP($\Gamma$), assign $\mu_v[t]$ to be $1$ if the variable $x_v$ takes value $t$ in the optimal solution to the CSP($\Gamma$) instance and $0$ otherwise.
	Furthermore, for each $C_i \in \mathcal{C}$ set $\lambda_{C_i}[y]$ equal to $1$ if $y$ is the local assignment for the scope $S_i$ corresponding to the optimal solution to the CSP and $0$ otherwise. 
	It follows that this is a feasible LP solution. 
	Hence, $OPT(\mathcal{C}) \le OPT_{LP}(\mathcal{C})$.
\end{proof}

\begin{remark}
It is worthwhile to note that conditions in the Basic LP do not ensure that two constraints having same scope have same values for the relaxed indicators $\lambda$ over different local assignments. This constraint can be imposed in the following manner.\\
For $C_{i}=(R_i, S_i)$ and $C_j=(R_j, S_j)$, if we have $S_i = S_j$, then
\begin{equation}\label{extracons}
\lambda_i[L] =   \lambda_j[L] ~ \ \forall L \in \mathcal L_i = \mathcal L_j
\end{equation}
However we ignore this condition owing to the difficulty that pops up in using them in a meaningful way while rounding up the fractional solution (to get a full assignment for the original CSP) that we get from the LP relaxation.  
\end{remark}

\subsection{LP as an Expectation Maximization}
Note that between constraints \ref{eq:canonLPmuNonNeg} and \ref{eq:canonLPmuSum}, a solution $\mu^*_v[\ell]$ to the above LP defines a probability distribution (over $\ell \in D$) of assignments for $v$. That is, if we wanted to randomly generate an assignment $\hat{\ell}_v$ for variable $v$, then we could simply state $\hat{\ell}_v = \ell$ with probability $\mu^*_v[\ell]$. Given this interpretation, we write
\begin{equation}
\mu_v[\ell] = \mathbb{P}\left( \hat{\ell}_v = \ell \right) \quad \text{ where } \quad \hat{\ell}_v \sim \mu_v 
\end{equation}

A similar interpretation holds for $\lambda$; between constraints \ref{eq:canonLPlambdaNonNeg} and \ref{eq:canonLPlambdaSum}, a solution $\lambda^*_i[L]$ defines a probability distribution (over $L \in \mathcal{L}_i$) of possible local assignments for constraint $C_i$. That is, if we wanted to randomly select a local  assignment $\hat{L}_i$ for constraint $C_i$, we could state $\hat{L}_i = L$ with probability $\lambda^*_i[L]$. Given this interpretation, we write
\begin{equation}
\lambda_i[L] = \mathbb{P}\left( \hat{L}_i = L \right) \quad \text{ where } \quad \hat{L}_i \sim \lambda_i 
\end{equation}
  
These distributions are tied together by constraint \ref{eq:canonLPConsistency}. In the probabilistic terms established above, constraint \ref{eq:canonLPConsistency} reads
\begin{equation}
\mathbb{P}\left( \hat{\ell}_v = \ell \right) = \sum_{\substack{L \in \mathcal{L}_i \\ L(v) = \ell}} \mathbb{P}\left( \hat{L}_i = L \right) \qquad \forall v \in V, \ell \in D, i : C_i \in C
\end{equation}
Since the events  $\{\hat{L}_i = L_1 \}$ and $\{\hat{L}_i = L_2 \}$ are mutually exclusive, the right hand side can be rewritten to give the following. This simply states that the probability that a variable $v$ takes on value $\ell$ is the same whether you consider the distribution as being defined from $\mu_v$ or $\lambda_i$, for any $i : C_i \in C$. We refer to this constraint as a \textit{first moment consistency constraint}.
\begin{equation} 
\mathbb{P}\left( \hat{\ell}_v = \ell \right) = \mathbb{P}\left( \bigcup_{L : L(v) =  \ell} \left\{\hat{L}_i = L\right\} \right) \qquad \forall v \in V, \ell \in D, i : C_i \in C
\end{equation}

As we remarked in Section \ref{sec:introToCSP} and Equation \ref{expfirst}, we can write the objective function of a CSP in the probabilistic terms $\max_{F} \mathbb{E}_{\tilde{C} \sim W}\left[R_F(\tilde{C})\right] $. With slight modifications (to reflect the fact that our decision variables are now \textit{local} rather than global assignments), we can make a similar statement. 

We define a new random variable $R(\tilde{C}, L(\tilde{C}))$, where $\tilde{C}$ is as defined before i.e., 
$$\mathbb{P} (\tilde{C} = C_i ) = w_i$$
and 
$$\mathbb{P}(L(\tilde{C}) = L |\tilde{C}=C_i ) =  \lambda_i[L]\  \forall L \in \mathcal{L}_i,$$ 
then we can express our objective in \ref{glbobj} as 
\begin{equation}
\max_\lambda \mathbb{E}_{\tilde{C}, L(\tilde{C})} R(\tilde{C}, L(\tilde{C}))
\end{equation}
Using tower property we can write this as 
\begin{equation}\label{tower}
\max_\lambda \mathbb{E}_{\tilde{C}} \left[ \mathbb{E} \left(R(\tilde{C}, L(\tilde{C}))\displaystyle|\tilde{C}\right). \right]
\end{equation}
To make the expression more legible we make some abuse of notation and write $C_i \sim W$ in place of $\tilde{C} \sim W$ and write $L \sim \lambda_i$ to denote that $L$ takes values in $\mathcal L_i$ according to $\lambda_i$ given $\tilde{C} = C_i$. And thus our objective is reduced to
\[
	OPT_{LP} = \max_{\lambda_{\mathcal{C}}} \underset{C_i \sim W}{\Ex} \left[ \underset{L \sim \lambda_{i}}{\Ex} \left[ R_i( L (S_i )) |C_i \right] \right]
\]
where $R_i( L (S_i ))$ is an indicator whether assignment $L$ for scope $S_i$ satisfies the relation given by $R_i$.
Having obtained a valid LP relaxation for any CSP problem, the question is now how to use this relaxation to generate good feasible solutions to the CSP. 
A key technique to generate CSP solutions is rounding the LP solution.


\subsection{A Rounding Scheme for the Canonical LP Relaxation}
Various techniques have been developed to extract good CSP solutions from Basic LP.
A common technique to generate CSP solutions with a performance guarantee is randomized rounding.
This technique uses the information available from the LP, typically by using the optimal solution as a probability distribution, to randomly round the each variable to a value in its domain $D$.
To showcase a potential rounding technique, let us consider the problem of Max $k$-SAT. 

\subsubsection{Randomized Rounding scheme for LP relaxation of Max k-SAT}\label{sec:lpRoundingSat}
This rounding scheme was introduced by Goemans and Williamson \cite{GoeWil94}. They combine this randomized rounding scheme together with the seminal approximation algorithm for Max-SAT by Johnson \cite{Joh73} to obtain a $\frac{3}{4}$ bound.
 
Consider the full assignment $F$ as a vector of random variables. For each variable $v \in V$, let the random variable be given by:
\[
	F(v) = \begin{cases}
	1 & \text{with probability } \mu^*_v[1]\\
	0 & \text{with probability } \mu^*_v[0]
	\end{cases}
\]
where $\mu^*_v[\ell]$ is the value of variable $\mu_v[\ell]$ in an optimal solution to the LP.
Since the sum over $\ell \in D = \{0,1\}$ of  $\mu_v[\ell]$ is $1$, and since $\mu_v[\ell]$ is non-negative, the above is a valid definition for a random variable. 
Furthermore, note that $F(1), \dots, F(n)$ is a feasible solution to the Max k-SAT problem.

The expected objective value of this randomized assignment is given by taking expectation in \ref{expfirst} and we get
\[
	\mathbb{E}_F  \left[\text{Val}_{\mathcal{C}}[F] \right]= \mathbb{E}_F \mathbb{E}_{\tilde{C}}\left[R_F(\tilde{C})\right]
\]
Since we can swap the order of expectation, this is equivalent to
\[
	 \mathbb{E}_{\tilde{C}} \mathbb{E}_F \left[R_F(\tilde{C})\right]
\]
and under our new notation we replace $\tilde{C}$ with $C_i$ to get 
\[
	 \mathbb{E}_{C_i \sim W} \mathbb{E}_F \left[R_F(C_i)\right].
\]
Now note that, since $R_F(C_i) = \mathbbm{1} [\mbox{$F$ satisfies $C_i$]}$ we get $\mathbb{E}_F \left[R_F(C_i)\right] = \mathbb{P}_F [\mbox{ $F$ satisfies $C_i$}]$ and our objective function becomes
\[
\mathbb{E}_F  \left[\text{Val}_{\mathcal{C}}[F] \right] = \underset{C_i \sim W}{\mathbb{E}}\left[ \underset{F}{\mathbb{P}}[F \text{ satisfies } C_i] \right]
\]
This swap allows us to consider each $C_i \in \mathcal{C}$ separately.
Therefore, a single constraint $C_i \in \mathcal{C}$ can be considered.

Observe that since the constraints are in disjunctive form, each literal needs to evaluate to false for the constraint to be false. 
In other words, there is a unique falsifying assignment $b_{C_i}$ that makes constraint $C_i$ false. 
For example, if $C_i = x_1 \vee x_2 \vee \bar{x}_3$, then the unique falsifying assignment is given by $x_1 = 0$, $x_2 = 0$, and $x_3 = 1$. 
Considering the definition of $F(v)$ and the independence of $F(i)$ and $F(j)$ for $i \neq j$, it then follows that 
\begin{equation}
		\underset{F}{\mathbb{P}}[ F \text{ satisfies } C_i] = 1 - \prod_{v \in S_i} \mu_v[b_{C_i}(v)] \label{eq:objectiveRounding}
\end{equation}
where $b_{C_i}(v)$ is the value of variable $v \in S_i$ in the falsifying assignment. 
In particular, the probability of not satisfying the constraint in the above example is $ 1- \mu_1^*[0] \mu_2^*[0] \mu_3^*[1]$.

Having identified the objective value of the randomized solution, the question is whether any guarantees can be made on the quality of the solution compared to the LP. 
This can be done, by relating $F$ to the distribution over local assignments.
\begin{align}
	p_{C_i} &:=  \underset{L \sim \lambda_{i}}\Ex \left[ R_i( L (S_i )) |C_i \right]\nonumber\\
			&= \underset{L \sim \lambda_{i}^*}{\mathbb{P}} [L \text{ satisfies } C_i] \nonumber\\
			&=  \underset{L \sim \lambda_{i}^*}{\mathbb{P}}\left[ \bigcup_{v \in S} \left\{ L(v) \neq b_{C_i}(v) \right\} \right] \nonumber\\
			&\le \sum_{v \in  S } \underset{L \sim \lambda_{i}^*}{\mathbb{P}} \left[  \left\{ L(v) \neq b_{C_i}(v) \right\} \right] \nonumber\\
			&= \sum_{v \in S} \left( 1 - \mu_v^*[b_{C_i}(v)] \right) \label{eq:objectiveLP}
\end{align}
This last step follows from the first-order consistency constraints:
\[
	\underset{L \sim \lambda_{i} }{\mathbb{P}}[ L(v) = \ell] = \underset{L \sim \mu_v }{\mathbb{P}}[ L = \ell]
\]

It remains to relate the LP objective value to the expected objective of the rounding procedure. 
Note that the former (see \eqref{eq:objectiveLP}) is an arithmetic mean (AM) whereas the latter (see \eqref{eq:objectiveRounding}) is an geometric mean (GM). 
This suggests the use of the inequality of arithmetic and geometric means.
\begin{align*}
		\underset{F}{\mathbb{P}}[ F \text{ satisfies } C_i] &= 1 - \prod_{v \in S_i} \mu_v[b_{C_i}(v)]\\
		&= 1 - \underset{v \in S_i}{GM}(\mu_{v}^*[b_c(v)])^{|S_i|}\\
		&\ge 1 - \underset{v \in S_i}{AM}(\mu_{v}^*[b_c(v)])^{|S_i|}\\
		&= 1 - \left( 1  - \underset{v \in S_i}{AM}(1 -\mu_{v}^*[b_c(v)]) \right)^{|S_i|}\\
		&\ge 1 - \left( 1  - \frac{p_{C_i}}{|S_i|}\right)^{|S_i|}\\
\end{align*}
One can show that $$1 - \left( 1  - \frac{p_{C_i}}{|S_i|}\right)^{|S_i|} \geq r \cdot p_{C_i} ~\ \forall p_{C_i} \in [0, 1]$$ and $r \downarrow 1-1/e$ as $|S_i| \rightarrow \infty$.Therefore, the objective value of the rounding scheme
\begin{align*}
	\underset{F}{\Ex} \left[ \text{Val}_{\mathcal{C}}[F]\right] &= \underset{F}{\Ex}\left[ \underset{C_i \sim W}{\mathbb{P}}[ F \text{ satisfies } C_i] \right]\\
	&=  \underset{C_i \sim W}{\mathbb{E}}\left[ \underset{F}{\mathbb{P}}[ F \text{ satisfies } C_i] \right]\\
	&\ge \left( 1 - \frac{1}{e} \right) \underset{C_i \sim W}{\Ex} [ p_{C_i} ]\\
	&= \left( 1 - \frac{1}{e} \right) \text{OPT}_{LP}(\mathcal{C})\\
	&\ge \left( 1 - \frac{1}{e} \right) \text{OPT}(\mathcal{C})\\
\end{align*}

Hence, we conclude with the following theorem:
\begin{thm}
	Randomized rounding is a $\left( ( 1 - 1/e)\beta, \beta \right)$-approximation for Max-SAT for any $\beta$.
\end{thm}
\subsection{CSP-Specific LP Relaxations and Rounding Schemes}
Note the following specific relaxations \cite{GoeWil94,Asa97,Yan94}.
\begin{itemize}
\item this thing
\item this thing as well
\item also this thing
\end{itemize}

	
	\newpage
	% sdp relaxation
	\section{Semidefinite Programming Relaxations for CSP's}

\subsection{Motivation : From ``First Order" to ``Second Order" Consistency Constraints}

We recall the relaxations and try to motivate our way through the various relaxations (see \cite{Ryan}) -

\begin{discussion}{\bf LP Relaxation}\\
In preparation for moving to the Basic SDP relaxation, we summarize (in words) the interpretations of $\lambda_i$ and $\mu_v$ in Basic LP.

For a CSP $\C = (V, C, W)$,  \textit{$\fa v \in V, \mu_v[\cdot] : D \rightarrow [0, 1]$ denotes the \textit{probability distribution of variable $v$ over its range $D$}. And, $\fa C_i \in C, \la_{C_i}[\cdot] : \Lm_{C_i} \rightarrow [0,1]$ denotes the probability distribution over all possible local assignments $L \in \Lm_{C_i}$ for variables in  $C_i$.  
Constraint \ref{eq:canonLPConsistency} imposes conditions for consistency across distribution the $\mu_v[\cdot]$ for a variable $v$ and its marginals with respect to any constraint distribution such that the scope of that constraint contains $v$.}
\end{discussion}

Constraint \ref{eq:canonLPConsistency} can be considered as first order constraint for variable assignment, where we are imposing consistency across marginals. 
We can go a step further in restricting our search space, by introducing a second order consistency condition such that we get an SDP relaxation. 
The relaxation will be with respect to IP formulation, but tightening with respect to the LP relaxation and hence our values will be now closer to the optimal value of CSP. 
The intuition for such a relaxation can be motivated by the relaxation procedure for MAX - CUT problem which we present next. 

It is further established, that this relaxation is in-fact the optimal poly-time relaxation for any general CSP, first by assuming Unique Games Conjecture \cite{Khot} and then later in fact more generally \cite{nphard}. 

\begin{example}{\bf MAX - CUT}\label{maxcut}\\
MAX CUT problem can be formulated as the following integer quadratic program : 
\al{
 \max \sum_{i,j} w_{ij} \ \frac{1}{2}(1-x_ix_j) ~:~ x_i \in \{ -1, 1\}
 }
Clearly $x_i^2=1$, and $x_i=x_j\rr x_ix_j=1$ and thus contribution of that term to the sum is zero. 
Also, $x_i \neq x_j \rr x_ix_j=-1$, and thus that weight contributes to the sum. 
We can see that by associating each vertex $i$ with the variable $x_i$, we have that if an edge has both end points with same $x$-value, it is ignored in the sum, and all the edges with end points in different groups contribute weight to the sum. 
\cite{delormecombinatorial} relaxed the above program by replacing each variable $x_i$ with a unit vector $y_i$ in $\R^n$ such that its norm is 1, i.e. $y_i \in S^{n-1}$, and replacing the $x_ix_j$ by the dot product between the vectors. 
\[ \max \sum_{i,j} w_{ij} \ \frac{1}{2}(1-y_i\trans y_j) ~:~ y_i\trans y_i = 1 {\mbox {\ i.e.\ }} y_i \in S^{n-1} \]
If we define a matrix $\Sigma$ such that $$\Sigma_{ij} = y_i\trans y_j,$$ then one can easily see that for $$Y = (y_1, \cdots, y_n) \rr \Sigma = Y\trans Y \succeq 0,$$ and our program reduces to 
\[ \max_\Sigma \sum_{i,j} w_{ij} \ \frac{1}{2}(1-\Sigma_{ij}) ~:~ \Sigma \succeq 0 ~:~ \Sigma_{ii} = 1  \]
which is clearly an SDP and can be solved in polynomial time, i.e. efficiently! 
Thus we see that the SDPs did pop out for a MAX CUT relaxation.
\end{example}

\subsection{Semi-Definite Program}
We define the semi-definite program and then discuss the probabilistic assignment interpretation, besides explicitly showing that we are talking about a relaxation.

\begin{definition}\textbf{Basic SDP} \\
Let $\mathcal{C} = (V,C,W)$ be a CSP over domain $D$. The Basic SDP for $\mathcal{C}$ is
\begin{alignat}{2}
\max ~&~ \sum_{i : C_i \in C} \sum_{L \in \mathcal{L}_i}   w_i\lambda_i[L] R_i(L) & \\
s.t. ~ & ~ \sum_{L \in \mathcal{L}_i} \lambda_i[L] = 1  & \forall i : C_i \in C \label{eq:SDPlambdaSum} \\
     ~ & ~ \Sigma_{\lrb{v, \ell},\lrb{v', \ell'}} = \sum_{\substack {L\in \mathcal{L}_{C_i}\\ L(v) = \ell, L(v')=\ell'} }\la_{C_i}[L] &~\forall v,v' \in V, \ell,\ell' \in D, i : C_i \in C \label{eq:SDPtiePSD}\\
     ~ & ~ 0 \leq \lambda_i[L] \leq 1  & \forall  i : C_i \in C, L \in \mathcal{L}_i  \label{eq:SDPlambdaNonNeg} \\ 
     ~& ~ \\
     ~ & ~ \Sigma \succeq 0& \label{eq:SDPPSD}
\end{alignat}
\end{definition}
\subsection{Basic SDP Relaxation - The Random Assignment Interpretation}

The SDP relaxation is similar to the LP relaxation but with an important generalization. 
We will have the exactly same probability distribution $\la_{C_i}$ for each constraint and the same objective function. 
We further imposed conditions on second order distribution of these variables. 
More precisely,
\al{
 \Pb_{L \sim \la_{C_i}}(L(v) = \ell, L(v')= \ell') = \Sigma_{\lrb{v, \ell},\lrb{v', \ell'}}, ~:~ \Sigma \succeq 0 \label{eq0}
 } 

We are ready to properly define the SDP relaxation using this random variable representation and its covariance structure.  
We denote an SDP solution by $\Sm$. Also, $R_i(L(S_i))$ is the indicator if the local assignment $L$ on $S_i$ satisfies the constraint $R_i$. This also gives: 
\al{
\Ex_{L \sim \la_{C_i}} \lrb{R_i(L(S_i))} = \Pb_{L \sim \la_{C_i}} (L:S_i \rightarrow D^{|S_i|} {\rm \ satisfies \ } R_i)
}
\begin{definition}{\label{def1}}
{\bf Basic SDP for CSP$(\C)$}  
\al{
\mbox{SDPOpt}(\C) &= \max_{\substack{\la_{C_i}, \Si, A}} \mbox{SDPVal}_\C(\Sm) := \Ex _{C_i \sim W} \lrbb{\Ex_{L \sim \la_{C_i}} \lrb{R_i(L(S_i))}} %\text{\footnotemark} \\
%&=  \max_{\substack{\la_{C_i}, \Si, A}} \sum_{C_i \in \C}w_{C_i} \Ex_{L \sim \la_{C_i}} \lrb{R_i(L(S_i))}
}
%\footnotetext{A more general objective function can be defined with respect to payoff  functions $\phi_C$ associated with each constraint: $$\max_{\la_C, C\in \mathcal{C}} \sum_{C=(R,S) \in \mathcal{C}} w_C \Ex_{L \sim \la_C} \phi_C(L) = \sum_{C \in \mathcal{C}} \sum_{L \in \mathcal{L}_C} \la_C(L) \lrb{w_C\phi_C(L)}  $$}

subject to the constraint $\fa C_i = (R_i, S_i) \in \mathcal{C},\fa v, v' \in S, \fa \ell, \ell' \in D $
\al{
\Pb_{L \sim \la_{C_i}} (L(v) = \ell, L(v') = \ell')  &\os{\lrb{\ref{eq0}}}= \Sigma_{\lrb{v, \ell},\lrb{v', \ell'}} = \Ex \lrbb{I_v(\ell) \cdot I_v(\ell')} \label{csm}\\
\Si &\succeq 0.
}
And we also impose the constraints for valid probability distribution : 
\al{
\sum_{L \in \Lm_i} \la_{C_i}(L) &= 1, \fa C_i \in \C\\
\sum_{\ell \in D}\Pb_{L \sim \la_{C_i}} (L(v) = \ell) &= 1, \fa v\in V, \fa C_i \in \C \label{eq2}\\
\Pb_{L \sim \la_{C_i}} (L(v) = \ell, L(v) = \ell') &= 0, \fa \ell \neq \ell', \fa v\in V, \fa C_i \in \C \label{eq1}
}
We also define the following constraint:\footnote{This condition is deemed redundant by Remark \ref{remark01}, but explicitly assuming it here makes our life easy.} 
\al{
\Pb_{L \sim \la_{C_i}} (L(v) = \ell ) = \Ex[I_v(\ell)] = A_{v, \ell} \label{cfm}
}
\end{definition}

\subsection{Is Definition \ref{def1} an SDP?}

Note that for a given $C_i =(R_i, S_i) \in \mathcal{C}$, $\la_{C_i}$ is a probability distribution over all possible assignments $\mathcal{L}_{C_i} = \{L: S \rightarrow D\}$. Clearly,  
\al{
\Ex_{L \sim \la_{C_i}} \lrb{R_i(L(S_i))} &= \Pb_{L \sim \la_{C_i}} (L:S_i \rightarrow D^{|S_i|} {\rm \ satisfies \ } R_i) \\
&= \sum_{L \in \mathcal{L}_{C_i}} \la_{C_i}(L) \mathbb{I} \lrbb{ L {\rm \ satisfies \ }R _i }
}
is linear in $\la_{C_i}$'s.
This makes the objective explicitly linear in the variables $\la_{C_i}$'s:
\al{
\Ex _{C_i \sim W} \lrbb{\Ex_{L \sim \la_{C_i}} \lrb{R_i(L(S_i))}} &= \sum_{C_i \in \C}w_{C_i} \Ex_{L \sim \la_{C_i}} \lrb{R_i(L(S_i))} \\
&=  \sum_{C_i \in \mathcal{C}} \sum_{L \in \mathcal{L}_{C_i}} \la_{C_i}(L) \lrb{w_{C_i}\mathbb{I} \lrbb{ L {\rm \ satisfies \ }R _i }} 
}
Next we see that 
\al{
\Sigma_{\lrb{v, \ell},\lrb{v', \ell'}} &=\Pb_{L \sim \la_{C_i}} (L(v) = \ell \wedge L(v) = \ell') \\
&= \sum_{\substack {L\in \mathcal{L}_{C_i}\\ L(v) = \ell, L(v')=\ell'} }\la_{C_i}(L)
}
is a linear constraint in the variables. Next, 
\al{
&&\sum_{\ell \in D}\Pb_{L \sim \la_{C_i}} (L(v) = \ell) &= 1 \\
&\lrr &\sum_{\ell \in D}  \sum_{\substack {L\in \mathcal{L}_{C_i}\\ L(v) = \ell }}\la_{C_i}(L) &= 1
}
is also linear. 
It is trivial to see that the last constraint is also linear:
\al{
&&\Pb_{L \sim \la_{C_i}} (L(v) = \ell ) &= \Ex[I_v(\ell)] = A_{v, \ell} \\
&\lrr   &\sum_{\substack {L\in \mathcal{L}_{C_i}\\ L(v) = \ell }}\la_{C_i}(L) &= A_{v, \ell} 
}
Last but not the least, we also have, 
\al{
\Si \succeq 0
}
which is a linear matrix inequality. Thus, with a linear objective and some linear constraints, besides an SDP constraint we do have an SDP relaxation.

\subsection{Is it a relaxation?}
To show that this is indeed a relaxation, we have to show that the optimal assignment belongs to our search space. 

Suppose that the optimal assignment was $v=\ell_v$, then we have to construct a positive semi-definite matrix $\Sigma$ such that choosing it gives us precisely the above assignment. 
Define the entries of $\Sigma$ as follows 
$$ \Sigma_{\lrb{v, \ell},\lrb{v', \ell'}} = \mathbbm{1}(\ell = \ell_v) \cdot \mathbbm{1}(\ell' = \ell_{v'}) = \mathbbm{1}\lrb{\ell = \ell_v, \ell'= \ell_{v'}}.$$ 

Note that \cite{lambdaandsigma} condition for this $\Sigma$ means that corresponding $\lambda_{C_i}$'s can be positive for only those local assignments $L$ which assign $\ell_v$ to $v$ for $v$ in their scope. 
For all other assignments, $\lambda_{C_i}[L]=0$.
Now we are left to show that such a $\Sigma$ is symmetric and positive semi-definite.

Clearly $\Sigma$ is symmetric. 
Also, 
\al{
y\trans \Sigma y &=\sum_{(v, \ell)} \sum_{(v', \ell')}  y_{(v, \ell)} \Sigma_{\lrb{v, \ell},\lrb{v', \ell'}} y_{(v', \ell')}  \\ 
&= \sum_{(v, \ell)} \sum_{(v', \ell')}  y_{(v, \ell)} \mathbbm{1}\lrb{\ell = \ell_v, \ell'= \ell_{v'}} y_{(v', \ell')}   \\
&=\sum_v \sum_{v'}  y_{(v, \ell)} y_{(v', \ell_{v'})}  \\
&= \sum_v y_{(v, \ell_v)}^2 + \sum_{v \neq v'} y_{(v, \ell)} y_{(v', \ell_{v'})} \\
&=  \lrb{\sum_{v}y_{(v, \ell_v)}}^2\\
&\geq 0
}
proving that $\Sigma$ is PSD. 

 Thus our search space does include the optimal assignment and hence this is indeed a relaxation.

\subsection{Another Interpretation}

There is a more popular but equivalent way to define these constraints, and it helps us to understand the rounding scheme that we will discuss later on. 
Note that to any positive semi-definite matrix $\Sigma$, we can associate $L$ such that, $\Si = LL\trans$, and in turn we can label $\Si$ as the covariance matrix of some joint random variables. 
Further, if we take these joint random variables as Gaussian, then we can as well make draws from them, as the first and second moments of a Gaussian  uniquely determine its distribution. 
These observations will be crucial for rounding process, but we stated them here to motivate this alternate expression of the constraints. 

We associate $\Si$ to joint real random variables $\lrb{I_v(\ell)}_{v\in V, \ell \in D}$, by calling it their covariance matrix. Since we impose $\Si \succeq 0$, it is indeed a valid representation. 
We will also have constraints which cause these random variables to hang together with the $\la_{C_i}$'s in a gentlemanly fashion. 
These joint random variables $I_v(\ell)$ will be called as \textit{pseduoindicator random variables}. 
We emphasize that they are jointly distributed. 


\subsection{Some Analysis}
\begin{remark}\label{remark01}
If we drop the ``consistent first moment" condition (\ref{cfm}) from the Basic SDP, we get
an equivalent formulation; i.e., any solution $\mathcal{S}$ to the Basic SDP which doesn't satisfy  (\ref{cfm}) can be transformed into a solution $\mathcal{S}'$ which does satisfy  (\ref{cfm}) and has $\mbox{SDPVal}_\C(\Sm')
= \mbox{SDPVal}_\C (\Sm)$.
\end{remark}

\begin{remark}
The consistent first and second moment constraints imply 
\al{
\Ex \lrbb{I_v(\ell)} = \Ex \lrbb{I_v(\ell)^2}
}
\end{remark}
\begin{proof}
\al{
\Ex \lrbb{I_v(\ell)^2)} &= \Pb_{L \sim \la_C} (L(v) = \ell \wedge L(v) = \ell) \\
&= \Pb_{L \sim \la_C} (L(v) = \ell ) \\
&= \Ex[I_v(\ell)] 
}
\end{proof}
\begin{remark}
\al{
 \sum_{\ell \in D} I_v(\ell) = 1 \quad \mbox{a.s.}, ~ \fa v \in V
 }
\end{remark}
\begin{proof}
Define $J_v = \sum_{\ell \in D} I_v(\ell)$, then 
\al{
\Ex[J_v] = \sum_{\ell \in D} \Ex [I_v(\ell)] =  \sum_{\ell \in D} \Pb_{L \sim \la_C} (L(v) = \ell ) = 1
}
\al{
\Ex[J_v^2] &= \Ex \lrbb{\lrb{ \sum_{\ell \in D} \Ex [I_v(\ell)]}\lrb{ \sum_{\ell' \in D} \Ex [I_v(\ell')]}} \\
&= \sum_{\ell, \ell' \in D} \Pb_{L \sim \la_C} (L(v) = \ell \wedge L(v) = \ell') \\
&\os{\lrb{\ref{eq1}}}= \sum_{\ell \in D}\Pb_{L \sim \la_C} (L(v) = \ell) \\
&\os{\lrb{\ref{eq2}}}=1\\
&=\Ex[J]^2
}
Thus we have 
\al{
\mbox{Var}(J_v) = \Ex[J_v^2] - \Ex[J_v]^2 = 1-1 = 0 \rr J_v = J  \equiv 1 \ \mbox{a.s.}
}
\end{proof}


	\newpage
	%rounding schemes
	\section{Rounding Schemes for SDP Relaxations}
\subsection{Intuition for Rounding}
Next question that comes to our mind is - 
\textit{``How do we get a near optimal assignment from the solution of the relaxed problem"}. 
Note that the solution of the relaxed problem gives us a PSD matrix and probability distributions over constraints. 
Next challenge is to make a ``consistent draw" from this distribution and analyzing the expected value of CSP resulting from such draws, 
however we have already understood the intuition behind various terms and hence it is merely a computational task now.  

We concretely summarize various angles, that we discussed briefly before, to look at canonical SDP relaxations for a CSP :
\iit{

\item Pseudo-indicator random variables which satisfy the first and second moment consistency constraints. 
This perspective is arguably the best for understanding the SDP. 

\item Pseudo-indicator random variables which satisfy the second moment constraints. 
This perspective is arguably the best when constructing SDP solutions by hand.

\item Vectors $\{ y_{v, \ell} \}$ satisfying the first and second moment consistency constraints.  
This perspective is the one that's actually used computationally, on a computer. 

\item Jointly Gaussian Pseudo-indicator random variables which satisfy the consistent first and second moment constraints. 
This perspective is suited for developing``SDP Rounding algorithms". 
}
Let's begin with a classical rounding scheme for MAX-CUT and and follow with a more general SDP rounding scheme for Unique Games. In the next section, we comment on some of the claimed optimal rounding schemes for SDP relaxations of any CSP.

\subsection{Goemans Williamson Algorithm}
By now, we know how to relax MAX-CUT to an SDP, and assuming efficient poly-time algorithm for SDP solver, we have a solution for the SDP formulation. Now, we want to somehow convert that back to \textit{good solution} for Max-cut. 

We will borrow notation and variables from Example \ref{maxcut}.
We assume that given $\Si$ we can find\footnote{We discuss this in appendix.} $Y : \Si = Y\trans Y$. 
Next we discuss the rounding discussed in \cite{gwFirstMaxCutSDP}. 

We want to cut the vectors $\{y_i\}_{i \in V}$ with a random hyper plane through origin such that all vectors on one side of he hyperplane correspond to variables in one partition, and the rest are classified into other partition, thereby giving us a cut. 

{\bf Algorithm}
\begin{enumerate}
\item Draw a vector $\hat{n} \in \R^n$ (where $\hat{n}$ denotes the normal to the hyperplane) from any rotationally symmetric distribution. 
\item Set $x_i = \sgn{\hat{n} \cdot y_i} \in \{-1, 1 \}$.
\end{enumerate}

{\bf Analysis of the Rounding} \\
For a given graph $G = (V, E)$, with $V =\{1, \ldots, n\}$, and $E$ the set of edges, fix $(i, j) \in E$. Then the probability that the edge $(i, j)$ is cut by the hyperplane is same as the probability that hyperplane splits $y_i$, and $y_j$. 
Now consider just simply the $2D$ plane containing $y_i, y_j$. Since the hyperplane was chosen from a rotationally symmetric distribution, the probability that it cuts these two vectors is same as that a random diameter in the circle containing $y_i, y_j$, lies in between the angle $\theta$ of these two vectors. Thus, 
\al{
\Pb[(i, j) \mbox{ gets cut} ] &= \frac{\theta}{\pi} \\
&=\frac{{\rm cos\inv}\lrb{y_i \cdot y_j}}{\pi}\\
&=\frac{{\rm cos\inv}\lrb{\Si_{ij}}}{\pi}\\
\Ex[\mbox{Weight of edges cut}] &= \sum_{(i, j) \in E}w_{ij} \frac{{\rm cos\inv}\lrb{\Si_{ij}}}{\pi}
}
Now recall that 
\al{
\mbox{SDPOpt} = \sum_{(i, j) \in E} w_{ij} \lrb{\frac{1}{2}-\frac{1}{2}\Si_{ij}} \geq \mbox{Opt}.
}
So, if we find $\alpha$ such that 
\al{
\frac{{\rm cos\inv}\lrb{\Si_{ij}}}{\pi} = \alpha\ \lrb{\frac{1}{2}-\frac{1}{2}\Si_{ij}} \fa \Si_{ij} \in [-1, 1]
}
then we can conclude 
\al{
\Ex[\mbox{cut val}]\geq \alpha \mbox{SDPOpt} \geq \alpha \mbox{Opt}
}
Solving it numerically we get that $\alpha = 0.87856$ works.
\begin{remark}
$\Ex[\mbox{Cut}]  \geq 0.87856 \mbox{\ SDPOpt} \geq 0.87856 \mbox{\ Opt }$
\end{remark} 
 
 \subsection{Gaussian Rounding}

We can see Unique Games as Constraint Satisfaction Problems that are a generalization of Max-Cut to a large domain size. Let us define an equivalent definition of Unique Games to make this relation precise:
\begin{definition}{\bf Unique Game}\\
A unique game consists of a constraint graph $G = (V, E)$, an assignment function $F: V \rightarrow D$ and a set of permutations $\pi_{vv'}$ on $D = \{0, \ldots, q-1 \}$ (for all edges $(v, v')$). Each permutations $\pi_{vv'}$ defines the constraint $\pi_{vv'}(F(v'))=F(v)$. The goal is to find an assignment $F$ soas to maximize the number of satisfied constraints.
\end{definition}
Thus, max-cut is a unique game with $q=2$ and vice-versa, unique games is a generalization of max-cut to larger domain size. 

Next we discuss the rounding scheme from \cite{cmm06}, which is a generalization of the Goemans Williamson Algorithm for the Max-Cut to a Rounding Algorithm for Unique Games. 

Given a Unique Game $G = (V,E)$ with edge weights $W = \{w_{vv'} | (v, v') \in E \}$, we formulate it as a CSP with constraints $\C$ and weights $W$. Then, we solve the SDP relaxation for this CSP instance and decompose the solution $\Sigma = U\trans U$ with $U \in \R^{N \times N}$ with $N = |V| \cdot |D|$, using Cholesky decomposition (which is polytime). 

Denote by $[x]_r$ the function that rounds $x$ up or down depending on whether the fractional part of $x$ is greater or less than $r$. 
Note that if $r$ is uniformly distributed in the interval $[0,1]$, then the expected value of $[x]_r$ is $x$. 

{\bf Algorithm}
\begin{enumerate}
\item Pick a number $r$ in the interval $[0, 1]$ uniformly at random.
\item Pick random independent Gaussian vectors $g_1, \ldots, g_{2q}$ with independent components distributed as $\mathcal{N}(0, 1)$.
\item For each vertex $v$:
\begin{enumerate}
\item Find normalized vectors $\tilde{u}_{\ell} = {u_{\ell}}/{||u_\ell||_2^2}$
\item Set $s_{u_\ell} = [2q \cdot ||u_\ell||_2^2]_r$, for $\ell=0, \ldots, q-1$.
\item For each $\ell$, project $s_{u_\ell}$ vectors $g_1, \ldots, g_{s_{u_\ell}}$ to $\tilde{u}_\ell$:
\al{
\xi_{u_\ell, s} = g_s\trans \tilde{u}_\ell, 1 \leq s \leq s_{u_\ell}.
}
Hence for each variable $u$, there are $s_{u_0}+\ldots+s_{u_{q-1}}$ many $\xi$'s, call this set $\Xi_u$.
\item For each $u$, find the maximum magnitude member in $\Xi_u$ and let it be $\xi_{u_{\ell^*}, s^*}$. Assign
\al{
F(u) = \ell^*
}
\end{enumerate}
\end{enumerate}

\begin{theorem}
If the optimal solution of the unique game satisfies $1-\epsilon$ fraction of constraints, then the above algorithm in expectation satisfied $1-O({\sqrt{\epsilon \log q }})$ fraction of constraints.
\end{theorem}
Proof of the above theorem is quite involved and we refer the reader to \cite{cmm06} for the analysis.

	\newpage
	\documentclass[letterpaper, 12pt]{article}

\usepackage[margin=2.5cm]{geometry}
\usepackage{placeins, graphicx}
\usepackage{amsmath,amsthm,amssymb}
\usepackage[]{mathtools}
\usepackage[]{bbm}

\numberwithin{equation}{section}

% --to donotes
\usepackage{xargs}                      % Use more than one optional parameter in a new commands
\usepackage[dvipsnames]{xcolor}
\usepackage[colorinlistoftodos,prependcaption,textsize=tiny]{todonotes}
\newcommandx{\unsure}[2][1=]{\todo[linecolor=red,backgroundcolor=red!25,bordercolor=red,#1]{#2}}
\newcommandx{\change}[2][1=]{\todo[linecolor=blue,backgroundcolor=blue!25,bordercolor=blue,#1]{#2}}
\newcommandx{\info}[2][1=]{\todo[linecolor=OliveGreen,backgroundcolor=OliveGreen!25,bordercolor=OliveGreen,#1]{#2}}
\newcommandx{\maybeinclude}[2][1=]{\todo[linecolor=Orange,backgroundcolor=Orange!25,bordercolor=Orange,#1]{#2}}
\newcommandx{\improvement}[2][1=]{\todo[linecolor=Plum,backgroundcolor=Plum!25,bordercolor=Plum,#1]{#2}}

\usepackage[]{thmtools}
\usepackage[dvipsnames]{xcolor}
\declaretheoremstyle[
	bodyfont=\normalfont, 
	spaceabove=0.5cm]{defFormat}
\declaretheoremstyle[
	postheadspace=1cm,
	bodyfont=\normalfont, 
	spaceabove=0.5cm]{inLineDefFormat}
\declaretheoremstyle[
	spaceabove=0.5cm, 
	spacebelow=0.5cm,
	postheadspace=1cm]{namedTheorem}
\declaretheorem[
	numberwithin=section, 
	style=defFormat, 
	shaded]{definition}
\declaretheorem[
	numbered=no, 
	style=defFormat, 
	shaded]{algorithm}
\declaretheorem[
	style=inLineDefFormat, 
	sibling=definition,
	shaded,
	name=Definition]{ILdefinition}
\declaretheorem[
	numbered=no, 
	style=namedTheorem, 
	shaded, 
	name=The Unique Games Conjecture]{ugc}
\newtheorem{thm}{Theorem}
\declaretheorem[numberwithin=section, style=defFormat, shaded]{Definition}
\begin{document}
\newpage
\section{``Optimal" SDP Rounding Schemes}\label{sec:optRoundSchemes}
In 2008, Prasad Raghavendra published a paper entitled \textit{Optimal Algorithms and Inapproximability Results for Every CSP?} (we will refer to the paper simply as \textit{Optimal Algorithms}). We highlight the important characteristics below:

\begin{itemize}
\item Raghavendra proposed an SDP-based rounding scheme for use on \textit{any} CSP.
\item The performance guarantees cannot be stated in the usual ``$\alpha$-approximation," or even ``$(\alpha,\beta)$-approximation" sense for $\alpha,\beta \in \mathbb{R}$..  These guarantees are called \textit{non-explicit}.
\item If the Unique Games Conjecture were true, then the proposed algorithm would be \textit{optimal} in that no polynomial time algorithm could provide stronger performance guarantees.
\item The proof is unusual in that if the UGC does not hold, then performance guarantees \textit{disappear} for CSP's with arity greater than 2.
\item The algorithm does have some performance guarantees for 2-CSP's irrespective of the truth of the UGC.
\end{itemize}

The third of the points above seems truly profound. It suggests that even purpose-built approximation algorithms could have asymptotically inferior performance to Raghavendra's generic algorithm. Indeed, many using optimization in practice would want to know: is Raghavendra's generic algorithm more computationally expensive than some purpose-built algorithms? Does Raghavendra's algorithm outperform purpose-built algorithms in practice? 

To add to these questions, Raghavendra and Steurer proposed yet another rounding scheme in their 2009 paper \textit{How to Round Any CSP}. Again, we highlight the important characteristics:

\begin{itemize}
\item They propose a generic SDP-based rounding scheme for use on any CSP, with accompanying performance guarantees that are independent of the truth of the UGC.
\item As before, the performance guarantees of the proposed algorithm are non-explicit
\item Their algorithm is ``polynomial time" and yet ``runs in time $\exp{(\exp{(\text{poly}(kq/\epsilon)}))}$".
\end{itemize}

The first point is at first very exciting, but the third is somewhat baffling. Where do these double-exponentials come from, and how is it that this algorithm could possibly be ``polynomial time?"

The remainder of this section is devoted to clearly communicating how these two algorithms would work if they were in fact implemented. Along the way, we will see the extent to which implementation is possible.
\subsection{Constructing, Solving, and ``Smoothing" $\text{SDP}_{\text{gen}}$}\footnote{In \textit{Optimal Algorithms ... for Every CSP?}, the following SDP was called SDP(I). In subsequent work (\textit{How to Round Any CSP}) the following SDP was called $\text{SDP}_{\text{gen}}$. We use the newer term here to prevent the misconception that these SDP's are different.
}
The goals of this section are to (1) develop a clear picture of the semidefinite program used in \textit{Optimal Algorithms} and \textit{How to Round Any CSP}, (2) explain an operation on the SDP solution (called ``smoothing") that both papers require as a technical detail, and (3) state the complexity of these steps combined. We will spend a fair amount of time on (1) before moving to (2) and (3).
\subsubsection{Constructing $\text{SDP}_{\text{gen}}$}
Coming to grips with the mathematical program ``$\text{SDP}_{\text{gen}}$" as stated in \textit{Optimal Algorithms} is complicated by the fact that it is \textit{not} an SDP as written. Convexity breaks down twice in ``$\text{SDP}_{\text{gen}}$".
\begin{enumerate}
\item ``$\text{SDP}_{\text{gen}}$" contains boolean variables $v_{i,c} \in \{0,1\}$
\item ``$\text{SDP}_{\text{gen}}$" contains equality constraints which are not affine.
\end{enumerate}
Nevertheless, the results of the paper are correct, as Basic SDP can be used in place of $\text{SDP}_{\text{gen}}$ throughout \textit{Optimal Algorithms} and \textit{How to Round Any CSP}. To make the algorithms in \textit{Optimal Algorithms} and \textit{How to Round Any CSP} accessible to the reader, we present a series of mathematical programs which achieve the goal intended by $\text{SDP}_{\text{gen}}$. Each of these programs use the following objective
\begin{equation}
\sum_{i : C_i \in C} w_i \sum_{L \in \mathcal{L}_i} P_i(L) \lambda_i[L]
\end{equation}
where $P_i(L)$ is a \textit{payoff function} taking a local assignment $L$ for constraint $i$ to the interval $[-1,1]$. This is a generalization of $R_i(L)$ as being a binary relation on $L \in \mathcal{L}_i$.

In \textit{Optimal Algorithms}, Raghavendra defines the variables $v_{(i,c)} \in  \{0,1\}$ as an indicator that variable $i \in V$ takes on value $c \in D$. Using this definition of $v$, the following constraints are intended by ``$\text{SDP}_{\text{gen}}$".
\begin{alignat}{2}
&\sum_{c \in D} v_{(i,c)} = 1 & \forall i \in V \label{eq:basicSDPIP_sumTo1} \\
&v_{(i,c)}v_{(i,c')} = 0        & \forall i \in V , c \in D, c' \neq c \label{eq:basicSDPIP_noContradictions} \\
&\sum_{\substack{L \in \mathcal{L}_i \\ L(i) = c \\ L(i') = c' }} \lambda_i[L] = v_{(i,c)}v_{(i',c')} & \forall i,i' \in V, c,c' \in D \label{eq:basicSDPIP_constrainLambda}
\end{alignat}
The above formulation is really a \textit{mixed integer geometric program} - an extraordinarily difficult to solve class of problems. Nevertheless, we may make one modification and take a big step toward Basic SDP. Namely, define $v_{(i,c)}$ not as indicators, but as \textit{vectors} in $\mathbb{R}^{N}_+$ for $N = n q$, $n = |V|$, and $q = |D|$.
\begin{alignat}{2}
(\ref{eq:basicSDPIP_sumTo1}) ~~\Rightarrow & ~~\sum_{c \in D} v_{(i,c)}^\intercal v_{(i,c)} = 1 & \forall i \in V \label{eq:basicSDPnearCVX_sumTo1} \\
(\ref{eq:basicSDPIP_noContradictions}) ~~\Rightarrow &~~v_{(i,c)}^\intercal v_{(i,c')} = 0 & \forall i \in V, c \in D, c' \neq c \label{eq:basicSDPnearCVX_noContradictions} \\
(\ref{eq:basicSDPIP_constrainLambda})~~ \Rightarrow & ~~\sum_{\substack{L \in \mathcal{L}_i \\ L(i) = c \\ L(i') = c' }} \lambda_i[L] = v_{(i,c)}^\intercal v_{(i',c')} &  i,i' \in V, c,c' \in D \label{eq:basicSDPnearCVX_constrainLambda}
\end{alignat}
The feasible region defined by the above constraint set is \textit{still} non-convex since we can add the constraints
\begin{equation}
v^k_{(i,c)} = 0 ~~ \forall ~~ k \in \{2,3,\ldots,N\} 
\end{equation}
and achieve an exact formulation for the CSP. To have a true convex mathematical program, we need to remove all non-affine equality constraints. Do this by defining
\begin{equation}
M = [v_{(1,0)},\ldots,v_{(1,q-1)},
	v_{(2,0)},\ldots,v_{(2,q-1)},\ldots,
	v_{(n,0)},\ldots,v_{(n,q-1)}] \quad  \in \mathbb{R}^{N \times N}
\end{equation}
as the Cholesky decomposition of a symmetric positive semidefinite matrix $\Sigma \in \mathbb{S}_+^{N \times N}$ (i.e. $\Sigma = M^\intercal M$). Now write the constraints using the fact that $\Sigma_{(i,c)(i',c')} = v_{(i,c)}^\intercal v_{(i',c')}$.
\begin{alignat}{2}
(\ref{eq:basicSDPnearCVX_sumTo1}) ~~\Rightarrow & ~~\sum_{c \in D} \Sigma_{(i,c)(i,c)} = 1 & \forall i \in V \label{eq:basicSDP_sumTo1} \\
(\ref{eq:basicSDPnearCVX_noContradictions}) ~~\Rightarrow &~~\Sigma_{(i,c)(i,c')} = 0 & \forall i \in V, c \in D, c' \neq c \label{eq:basicSDP_noContradictions} \\
(\ref{eq:basicSDPnearCVX_constrainLambda})~~ \Rightarrow & ~~\sum_{\substack{L \in \mathcal{L}_i \\ L(i) = c \\ L(i') = c' }} \lambda_i[L] =\Sigma_{(i,c)(i',c')} &  i,i' \in V, c,c' \in D \label{eq:basicSDP_constrainLambda}
\end{alignat}
The reader may refer to Definition \ref{def:BasicSDP} and note that (\ref{eq:basicSDP_sumTo1}) through (\ref{eq:basicSDP_constrainLambda}) along with $\Sigma \in \mathbb{S}^N_+$ and $0 \leq \lambda_i[L] \leq 1$ are precisely the constraints found in Basic SDP. It follows that $\text{SDP}_{\text{gen}}$ can \textit{effectively} be solved in polynomial time simply by solving Basic SDP and performing a Cholesky decomposition of $\Sigma = M^\intercal M$ (noting that the columns of $M$ are the desired vectors $v_{(i,c)}$). We are now ready to move on to the ``smoothing" operation mentioned at the beginning of this section. 

\subsubsection{Smoothing a Solution to $\text{SDP}_{\text{gen}}$}
As a technical detail, the rounding scheme in \textit{Optimal Algorithms} requires that SDP solutions be ``$\alpha$-smooth" (defined below). The rounding scheme from \textit{How to Round Any CSP} differs in that smoothing is required in its proof of correctness, but \textit{not} during run time. 

A solution $(M,\lambda)$ is said to be $\alpha$-smooth for some $\alpha >0 $ if:
\begin{equation}
\min_{\substack{ i : C_i \in C \\ L \in \mathcal{L}_i }} \lambda_i[L] \geq \alpha.
\end{equation}
Note that smoothness is defined with respect to $\lambda$, not $M$. Note also that an $\alpha$-smooth solution precludes the possibility of some $\lambda_i[L] = 0$, hence an optimal solution $(M, \lambda)$ is not-necessarily $\alpha$-smooth. To handle this, Raghavendra (and Stuerer, in \textit{How to Round Any CSP}) propose the following smoothing operation. If using the algorithm in \textit{Optimal Algorithms}, set $\alpha = \eta / 100$ where $\eta \ll 1$ the controls the error of the approximation algorithm.

 (The reader is advised that the smoothing procedure is not the focus of either of the algorithms that follow; one would do well to resolve confusion with the smoothing procedure after having read the rest of this section.)

\begin{algorithm}\textbf{Smooth$(\mathcal{C},M,\lambda, \alpha)$} \\

Input: The CSP $\mathcal{C}$, an optimal solution $(M,\lambda)$ to $\text{SDP}_{\text{gen}}(\mathcal{C})$, and a parameter $\alpha > 0$. 

Output : an $\alpha$-smooth solution to $\text{SDP}_{\text{gen}}(\mathcal{C})$.
\begin{itemize}
\item For each $i \in V$ and $c \in D$,
	\begin{itemize}
	\item Construct a partial SDP solution $u_{(i,c)}$ corresponding to a uniform distribution over all integral
	 solutions to
	$\text{SDP}_{\text{gen}}$ (a solution is said to be integral if all variables take on $\{0,1\}$ values).
	\item Reassign the vector-valued variable $v_{(i,c)} \leftarrow \sqrt{1-\alpha q^k} v_{(i,c)} \oplus \sqrt{aq^k}
	u_{(i,c)}$, where $\oplus$ is the external direct sum (or ``concatenation") operator. We note that before this step, we had	
	$v_{(i,c)} \in \mathbb{R}^N$, while after this step, we have $v_{(i,c)} \in \mathbb{R}^{2N}$.
	\end{itemize}
\item $\left[\text{Optional}\right]$\footnote{This step is optional in that although smoothness is defined with respect to $\lambda$, subsequent computations only use $v_{(i,c)}$} For each $i : C_i \in C$ and $L \in \mathcal{L}_i$,
	\begin{itemize}
	\item  Construct $\mu_i[L]$ as the remaining part of the uniform SDP solution.
	\item Reassign the variable $\lambda_i[L] \leftarrow (1-\alpha q^k)\lambda_{i}[L] + (\alpha q^k)\mu_i[L]$. 
	\end{itemize}
\end{itemize}
\end{algorithm}

\subsubsection{Time Complexity for Constructing, Solving, and Smoothing}\label{subsec:buildSolveSmooth}
Both algorithms require building and solve Basic SDP (as Basic SDP followed by a Cholesky decomposition is the natural convex representation of $\text{SDP}_{\text{gen}}$), as well as smoothing the resultant solution. In this section we demonstrate the complexity associated with this \textit{portion} of these algorithms. For the sake of clarity, we redefine some of our parameters.

Let $\mathcal{C} = (V,C,W)$ be fixed. Define the following parameters: $n = |V|$, $m = |C|$, $q = |D|$, and $ k = \max_{1\leq i \leq m}\text{ar}(P_i)$ (the arity of $\mathcal{C}$).

Given this, Basic SDP has $O(n^2q^2 + mk^q)$ variables and $ O(m (n^2q^2 + k^q))$ constraints. If we let $T^{SDP}\left[{n',m',\epsilon'}\right]$ denote the complexity of solving an SDP with $n'$ variables over $m'$ constraints to precision $\epsilon'$, then the time to construct and solve the SDP is approximately

\begin{equation}
O( T^{SDP}\left[n^2q^2 + mk^q,~ m (n^2q^2 + k^q),~\epsilon'\right])
\end{equation}

After solving the SDP, we have a matrix $\Sigma$, but we need a collection of vectors $\{v_{(i,c)}\}$. We get this collection of vectors by performing a Cholesky decomposition on $\Sigma = M^\intercal M$ to get a matrix $M$ whose columns are $\{v_{(i,c)}\}$. As stated in Section \ref{sec:sdp}, this can be accomplished with a naive $O(N^3)$ implementation. Regardless of whether or not faster algorithms are used, this does not change the asymptotic complexity built up so far.

Note that for since $\Sigma$ is inevitably positive \textit{semi-definite}, we cannot compute the Cholesky decomposition in a strict sense. Nevertheless, we can compute the $LDL^\intercal$ decomposition and merge $\sqrt{D}$ into the trianglular matrices to construct the desired $M$ (this process is equivalent to the Cholesky decomposition when $\Sigma \succ 0$). Since we can compute the $LDL^\intercal$ decomposition in approximately $O(N^2) \in O(n^2q^2)$ time, we have the desired $M$ in $O(n^2q^2)$ time. Luckily, this complexity is already present in the time to construct and solve Basic SDP.


Once we have $M$ and $\lambda$, we may need to construct a ``smooth" SDP solution.\footnote{As mentioned in Section \ref{subsec:buildSolveSmooth}, the smoothing is required during run time if we use the rounding scheme from \textit{Optimal Algorithms}, but smoothing is only used in a \textit{proof} if we are using the rounding scheme from \textit{How to Round Any CSP}.}  Note that by symmetry, we only need to construct the uniform solution $u_{(i,c)}$ for a single $i \in V$ and $c \in D$. Let $T^u$ denote the time required to construct this solution. If we assume that the smoothed vectors $v_{(i,c)} \in \mathbb{R}^{2N}$ need to be allocated in memory from scratch, then time to smooth is approximately $O(T^u + nq N) \in O(T^u + n^2q^2)$.

And so to get to the point just prior to rounding for these algorithms, we have the following.
\begin{align}
&\textit{How to Round Any CSP} \qquad O( T^{SDP}\left[n^2q^2 + mk^q,~ m (n^2q^2 + k^q),~\epsilon'\right])  \\
&\textit{Optimal Algorithms} \qquad O( T^{SDP}\left[n^2q^2 + mk^q,~ m (n^2q^2 + k^q),~\epsilon'\right] + T^u)
\end{align} 


\subsection{Optimal Algorithms ... for Every CSP? (2008)}
This section is organized as follows :
\begin{itemize}
\item First, we present the algorithm at a high level. Presenting the algorithm concretely in this way serves as a reference for the reader and a guidepost for understanding some of the more difficult aspects of the algorithm.
\item Second, we address key steps in detail, in the order that they would be executed during run time.
\item We conclude with some discussion on the practicality of the proposed approximation algorithm.
\end{itemize}

This established, we move on to the algorithm itself. In the original paper, the following algorithm is only known as ``Round." We give it a proper\footnote{although not necessarily \textit{good}} name out of necessity.

\begin{algorithm} \textbf{UGDFS} : \textbf{U}nique \textbf{G}ames \textbf{D}ependant \textbf{F}unction \textbf{S}earch \\

\textit{Input: } An instance $\mathcal{C}$ of $\text{CSP}(\Gamma)$, as well as parameters $\kappa$ and $\eta$.

\textit{Output: } An assignment $F$ of variables $V \in \mathcal{C}$.
\begin{itemize}
\item Build an SDP $\text{SDP}_{\text{gen}}(\mathcal{C})$.
\item Solve $(\Sigma,\lambda) \leftarrow \text{SDP}_{\text{gen}}(\mathcal{C}).\text{solve}$.
\item Perform a Cholesky decomposition on $\Sigma$ to find $M : M^\intercal M = \Sigma$. Identify the $(i,c)^{\text{th}}$ column of $M$ with the vector $v_{(i,c)}$.
\item Smooth the SDP solution $(M,\lambda) \leftarrow \text{Smooth}(\mathcal{C},M,\lambda, \alpha)$.
\item Discretize a certain space of functions ``$\Omega$" to form $\mathcal{S}_{\kappa}$.
\item For every function $\mathcal{F} \in \mathcal{S}_{\kappa}$, run a subroutine called ``$\text{Round}_{\mathcal{F}}(\mathcal{C},v)$" to get an assignment $F$ of the variables in $\mathcal{C}$. 
\item Return the best assignment generated over all of these $\mathcal{F} \in \mathcal{S}_{\kappa}$.
\end{itemize}
\end{algorithm}

\subsubsection{What is $\Omega$? How Do We Discretize It?}\label{subsubsec:UGDFS_discretize}
We need to define a few terms before we precisely state $\Omega$. Let $\Delta_q$ denote the set of standard basis vectors in $\mathbb{R}^q$, and define $\text{Conv}(\Delta_q)$ as the convex hull of these vectors. We plot $\text{Conv}(\Delta_2)$ and $\text{Conv}(\Delta_3)$ below.

\begin{center}
\includegraphics[scale=0.4]{../images/convDelta2} \qquad
\includegraphics[scale=0.4]{../images/convDelta3}
\end{center}

That is, $\text{Conv}(\Delta_2)$ is a line, and $\text{Conv}(\Delta_3)$ is a triangular segment of a plane. More generally, $\text{Conv}(\Delta_q)$ is the $q-$dimensional simplex.

For $q = |D|$, and some constant $R$ that is independent of the number of variables and constraints in $\mathcal{C}$,\footnote{$R$ does vary with other parameters. It depends on $q$, $k$ (the arity of $\mathcal{C}$), $\eta$ (the additive error desired in the approximation), and (most importantly) the Unique Games Conjecture. We describe $R$ concretely at the end of this section.}  define $\Omega$ as
\begin{equation}
\Omega \doteq \{ \mathcal{F} : D^R \to \text{Conv}(\Delta_q) \}.
\end{equation}
In words, $\Omega$ is the set of \textit{all functions} from $D^R$ to $\text{Conv}(\Delta_q)$.

As we will see later on, the distribution of possible assignments returned by $\text{Round}_{\mathcal{F}}$ changes with $\mathcal{F}$. In \textit{Optimal Algorithms}, Raghavendra proves that there always exists a function $\mathcal{F}^* \in \Omega$ such that the value of the assignment returned by $\text{Round}_{\mathcal{F}^*}$ is good in expectation. The trouble is that to find this $\mathcal{F}^*$, we have to search over \textit{all} of $\Omega$.

Of course, there are infinitely many functions in $\Omega$, so we cannot do exactly this. In order to proceed, we need to build a sufficiently large set of representative functions so that we can ensure we come \textit{close} to  $\mathcal{F}^*$.

The most conservative measure for the distance between two functions is the ``infinity norm" (defined below).
\begin{equation}
\| \hat{\mathcal{F}} - \mathcal{F} \|_{\infty} \doteq 
	\max_{\substack{\sigma \in D^R \\ i \in \{1,\ldots,R \}}} \left\{\left|\left[\hat{\mathcal{F}}(\sigma)\right]_i - \left[\mathcal{F}(\sigma)\right]_i\right|\right\}
\end{equation}

And so if we want to ensure that we miss $\mathcal{F}^*$ by no more than some small amount (say, $\kappa$), we need to build a set of functions $\mathcal{S}_\kappa$ such that for every $\mathcal{F} \in \Omega$, there exists some $\hat{\mathcal{F}} \in \mathcal{S}_\kappa$ such that $\| \hat{\mathcal{F}} - \mathcal{F} \|_{\infty} \leq \kappa $.

How can we construct $\mathcal{S}_\kappa$? We need only discretize the \textit{output space} $\text{Conv}(\Delta_q)$. For boolean CSP's, we would divide the line $\text{Conv}(\Delta_2)$ into $\lceil \sqrt{2}/\kappa \rceil$ intervals of length no more than $\kappa$. Alternatively for $ q = 3$, we partition $\text{Conv}(\Delta_3)$ into small squares of area approximately $1/\kappa^2$. Both of these instances are illustrated below.

\begin{center}
\includegraphics[scale=0.4]{../images/GConvDelta2} \qquad
\includegraphics[scale=0.4]{../images/GConvDelta3}
\end{center}

For larger $q$, the process becomes more complicated, but is roughly the same in principle.

One practical question remains - how many functions are in $\mathcal{S}_\kappa$? Remember that discretizing the \textit{output} space of $\mathcal{F} \in \Omega$ is only part of the battle. Note that for two finite sets $A$ and $B$, there are exactly $|B|^{|A|}$ unique functions from $A$ to $B$. Since we can expect something on the order of $\kappa^{-q}$ elements in the discretized versions of $\text{Conv}(\Delta_q)$, it follows that the cardinality of $\mathcal{S}_\kappa$ is on the order of
\begin{equation}
|\mathcal{S}_\kappa| \approx \left[\frac{1}{\kappa^q}\right]^{q^R}
\end{equation}

\subsubsection{The $\text{Round}_{\mathcal{F}}$ Subroutine}

\begin{algorithm}\textbf{ $\text{Round}_{\mathcal{F}}(\mathcal{C},M)$} \\

\textit{Input} : The CSP $\mathcal{C}$ and a matrix $M$ whose columns are the smoothed vectors $v_{(i,c)}$.

\textit{Output} : An assignment of variables $F : V \to D$. 
\begin{itemize}
\item Sample $R$ vectors $\psi^{(1)},\ldots,\psi^{(R)}$ in $\mathbb{R}^{2N}$ with each coordinate being i.i.d $\mathcal{N}(0,1)$. We note that these vectors are in $\mathbb{R}^{2N}$ because they must be the same length as smoothed $v_{(i,c)}$ (which are themselves in $\mathbb{R}^{2N}$).
\item For each variable $i \in V$,
	\begin{itemize}
	\item Compute $R$ projections $v_{(i,c)} \to \mathbb{R}$ in the following way
		\begin{equation}
			h^{(j)}_{(i,c)} = \mathbbm{1}^\intercal v_{(i,c)} 
				+ (1-\epsilon) \left[ \left(v_{(i,c)} - \left(\mathbbm{1}^\intercal v_{(i,c)}
				\right)\mathbbm{1}\right)^\intercal \psi^{(j)}\right]  \quad 1 \leq j \leq R
		\end{equation}
	\item Construct a vector $p_i \in \mathbb{R}^q$ in the following way.
		\begin{equation}
			p_i = \sum_{\sigma \in D^R} \left( \prod_{j = 1}^R h^{(j)}_{(i,\sigma_j)} \right) \mathcal{F}(\sigma)
		\end{equation}
		Where we note that $\mathcal{F}(\sigma) \in \text{Conv}(\Delta_q)$ and 
		$\left( \prod_{j = 1}^R h^{(j)}_{(i,\sigma_j)} \right) \in \mathbb{R}$.
	\item ``Round" $p_i$ to $p_i^* \in \text{Conv}(\Delta_q)$ using the following procedure (where $e_1$ is the first standard basis vector in $\mathbb{R}^q$).
		\begin{align}
			&x \in \mathbb{R} ~~~ f(x) \doteq \begin{cases} 0 & \text{ if } x < 0 \\ 
									x & \text{ if } 0 \leq x \leq 1 \\
									1 & \text{ if } x > 1 
						\end{cases} \\
			&x \in \mathbb{R}^q ~~~ g(x) \doteq 
								\begin{cases}
								x/\mathbbm{1}^\intercal x &\text{ if } \mathbbm{1}^\intercal x \neq 0 \\
								e_1 						 &\text{ if } \mathbbm{1}^\intercal x = 0
								\end{cases} \\
			& p_i^* \doteq g(f(p_{i0}),\ldots,f(p_{i,q-1}))
		\end{align}
	\item Assign variable $i \in V$ the value $j \in D$ with probability $p^*_{ij}$.
	\end{itemize}
\end{itemize}
\end{algorithm}

\subsubsection{Practicality of UGDFS}
As demonstrated in Section \ref{subsec:buildSolveSmooth}, the time complexity of getting to the beginning of discretizing $\Omega$ is $O( T^{SDP}\left[n^2q^2 + mk^q,~ m (n^2q^2 + k^q),~\epsilon'\right] + T^u + n^2q^2)$.

As we will see in this section, the time to discretize $\Omega$ and search over $\mathcal{S}_{\kappa}$ makes the algorithm prohibitively slow in practice. We showed in Section \ref{subsubsec:UGDFS_discretize} that the size of the set $S_{\kappa}$ is approximately $\left(\kappa^{-q}\right)^{q^R}$. Even if it were possible to identify each function in $O(1)$ time, we would still be looking at $O(\left(\kappa^{-q}\right)^{q^R})$ iterations of $\text{Round}_{\mathcal{F}}$. 


To put this in perspective, consider CSP's with domain size 3. For such CSP's set $\kappa = 1/3$ and assume $R = 2$ (the smallest possible $R$ that could possibly be consistent with its definition). From these parameters we have $|\mathcal{S}_\kappa| \approx 7.65\cdot 10^{12}$ (for comparison, $2^{32} \approx 4.29\cdot 10^9$). And so even if $\text{Round}_{\mathcal{F}}$ ran in $O(1)$ time per call, implementing UGDFS would be very impractical.

Large constants aside, there is another obstacle to implementing UGDFS : the constant ``$R$".

\begin{definition}\textbf{The Constant $R$ Used in UGDFS}\\
For $\mathcal{C} = (V,C,W)$ and error parameter $\eta$, set $\alpha = \eta/(100q^k)$, $\epsilon = \eta/(100k)$, and denote by $\tau$ a constant given by the Invariance Principle\footnote{A generalization of the central limit theorem considered in literature on CSP approximation.} that is itself a function of $\epsilon$ and $\alpha$. Set $\delta$ as follows.
\begin{equation}
\delta = \frac{\tau \eta^2}{3\cdot 10^4} \left(k e \tau \ln(1/(1-\epsilon)) \right)^{-2}
\end{equation}
Then $R$ is any integer large enough so that it is NP-Hard to distinguish between $ 
[\leq \delta]$-satisfiable and $[\geq (1-\delta)]$-satisfiable instances of bipartite Unique Label Cover over an alphabet of size $R$.
\end{definition}

$R$'s peculiar definition is the reason for the statement ``if the UGC does not hold, then performance guarantees \textit{disappear} for CSP's with arity greater than 2" in the beginning of Section \ref{sec:optRoundSchemes}. That is, if the UGC is false, then there exist $\delta,\eta$ so that appropriate $R$ \textit{does not exist}.\footnote{Except, as we noted, for CSP's of arity 2. We refer the interested reader to Raghavendra's paper.} 

Because at this point it has become clear that UGDFS is a \textit{theoretical} result, we opt not to derive the asymptotic run time of $\text{Round}_{\mathcal{F}}$.

\subsection{How to Round Any CSP (2009)}
This section is organized in the same way as the one prior.
\begin{itemize}
\item First, we present the algorithm at a high level.
\item Second, we explain the key steps in detail.
\item Finally, we discuss time complexity and performance guarantees.
\end{itemize}

\begin{algorithm} \textbf{The Variable Folding Method} \\

\textit{Input: } An instance $\mathcal{C}$ of $\text{CSP}(\Gamma)$, as well as a parameter $\epsilon$.

\textit{Output: } An assignment $F : V \to D$ of variables in $\mathcal{C}$.
\begin{itemize}
\item Build $\text{SDP}_{\text{gen}}(\mathcal{C})$.
\item Solve $(\Sigma,\lambda) \leftarrow \text{SDP}_{\text{gen}}(\mathcal{C}).\text{solve}$.
\item Perform a Cholesky decomposition on $\Sigma$ to find $M : M^\intercal M = \Sigma$. Identify the $(i,c)^{\text{th}}$ column of $M$ with the vector $v_{(i,c)}$.
\item Define vectors $u_{(i,c)}$ by projecting $v_{(i,c)}$ on to a random subspace of dimension $\beta$.[\footnote{The parameter $\beta$ is specified in the following subsections.}]
\item Identify a set of ``bad" SDP constraints by examining $\mathcal{C}$ and $(u_{(i,c)},\lambda_i[L])$. Remove the associated payoffs from $\mathcal{C}$ to define a new CSP $\mathcal{C}'$.
\item Define yet another CSP, ``$\mathcal{C}'/\phi$" $\leftarrow \text{Folding}(\mathcal{C}',\{u_{(i,c)}\},\epsilon)$.[\footnote{The variable set of  $\mathcal{C}'/\phi$ is bounded by a constant independent of the number of variables and constraints in $\mathcal{C}$. The constraint set is no larger than that of $\mathcal{C}'$.}]
\item Find an \textit{\textbf{optimal}} variable assignment $F^*$ for $\mathcal{C}'/\phi$ with any \textit{\textbf{exact}} algorithm.
\item Unfold $F^*$ to construct $F^{**}$ - a variable assignment for $\mathcal{C}$.
\end{itemize}
\end{algorithm}

\subsubsection{Dimension Reduction by Random Projection}
In the this rounding scheme's proof of correctness, it is required that $ \beta \gg k^2q^2/\epsilon^3$ ($\beta$ is otherwise arbitrary). Critically, the the bound on $\beta$ is constant with respect to the number of constraints and variables in $\mathcal{C}$. 

Once $\beta$ is selected, this step of the algorithm is quite simple: construct a matrix $\Phi \in \mathbb{R}^{\beta \times nq}$ where each entry is an independent and identically distributed Gaussian random variable with mean 0 and variance $1/\beta$. Given $\Phi$, simply set $u_{(i,c)} \doteq \Phi v_{(i,c)}$.

\subsubsection{Identifying ``Bad" Constraints}
The proof of correctness for the Variable Folding Method relies on the fact that the operations of \textit{dimension reduction} (which give us $u_{(i,c)} \in \mathbb{R}^{\beta}$) and \textit{folding} (defined later) results in a CSP $\mathcal{C}'/\phi$ with optimal SDP objective $\text{SDP}_{\text{Opt}}(\mathcal{C}'/\phi) \approx \text{SDP}_{\text{Opt}}(\mathcal{C})$. In particular, the proof constructs a solution for $\text{SDP}_{\text{gen}}(\mathcal{C}'/\phi)$ (with value approximately equal to $ \text{SDP}_{\text{Opt}}(\mathcal{C})$) with the help of the projections $u_{(i,c)}$. 

Note that if we simply ``plug in" $u_{(i,c)}$ and $\lambda_i[L]$ to the SDP constriants, we will find that some are violated. Since the constructive proof requires a \textit{feasible} solution to the SDP for $\mathcal{C}'/\phi$, we cannot leave these violated constraints unaddressed. A key technical insight of Raghavendra and Stuerer is how to address these violated constraints. They accomplish this in two ways.

\begin{itemize}
\item If a constraint is violated by less than $\epsilon$, we can make minute modifications to variables $(u_{(i,c)},\lambda_i[L])$ and manage to satisfy the constraint.
\item If a constraint is violated by more than $\epsilon$, it is hopeless and must be cast aside. Amazingly, no more than an $\epsilon$-fraction of the constraints are violated by more than $\epsilon$! And so we introduce a very small error when dropping the associated payoffs.
\end{itemize}

And so in terms of $\text{SDP}_{\text{gen}}(\mathcal{C})$, ``bad constraints" are those such that plugging in $u_{(i,c)}$ and $\lambda_i[L]$ results in violations greater than $\epsilon$. We emphasize though, that we discard not just constraints in the SDP (which are not intrinsic to the CSP $\mathcal{C}$), but also the \textit{payoff} $P_j$ associated with the violated SDP constraint. 

\subsubsection{Constructing $\mathcal{C}'/\phi$ by ``Folding"}

The algorithm that follows can be broken down into three phases. First, discretize the vectors $u_{(i,c)}$ by identifying each $u_{(i,c)}$ with the closest member $w_{(i,c)}$ of an ``$\epsilon$-net" over the unit ball in $\mathbb{R}^\beta$.\footnote{This discrete set has cardinality $\leq \left(\frac{\alpha}{\epsilon}\right)^{\beta}$ for some absolute constant $\alpha$.} 
Discretizing the vectors naturally divides them into equivalence classes. The second step of the algorithm is to use these equivalence classes on vectors to cluster \textit{variables}. 
At a high level, $i,i' \in V$ are merged into the same cluster iff $\phi(i) = \phi(i')$ where $\phi(i) \doteq [w_{(i,0)},\ldots,w_{(i,q-1)}]$ (i.e. if for \textit{ every } $c\in D$ we have $w_{(i,c)} = w_{(i',c)}$). 
The third step of the algorithm is to construct a new CSP $\mathcal{C}'/\phi$ with a variable set $Z \subset V$ and a payoff set appropriately derived from $Z$ and $\mathcal{C}'$.

We now describe this procedure in detail.

\begin{algorithm} \textbf{ Folding($\mathcal{C}',~\{u_{(i,c)}\}, ~\epsilon$)} \\

\textit{Input: } A CSP $\mathcal{C}'$, a set of projections $\{u_{(i,c)}\}$ of the vectors $\{v_{(i,c)}\}$, and a parameter $\epsilon$. 

\textit{Output: } A CSP $\mathcal{C}'/\phi$ whose variable set is bounded by a constant.

\begin{itemize}
\item Construct an $\epsilon$-net $H$[\footnote{Raghavendra and Stuerer appeal to a simple bound $|H| \leq \left(\frac{\alpha}{\epsilon}\right)^\beta$ for some absolute constant $\alpha$.}] of the unit ball in $\mathbb{R}^\beta$.
\item Initialize two empty ``map" data structures $A$ and $\phi$.
\item For each $i \in V$, 
	\begin{itemize}
	\item For each $c \in D$, determine closet representative of $u_{(i,c)}$ in $H$;
		denote these representatives $w_{(i,c)}$. 
	\item Assign $z \in \mathbb{R}^{\beta \times q}$ the value $ [w_{(i,0)},  \ldots, w_{(i,q-1)}]$.
	\item If $A$ does not contain a key equal to $z$,[\footnote{Where two matrices are equal if all entry-by-entry comparisons evaluate to ``true."}] create $S \leftarrow \{i\}$ and map the key $z$ to the set $S$. 
		If otherwise, retrieve and update $S \leftarrow S\cup\{i\}$.
	\end{itemize}
\item Initialize an empty variable set $Z$.
\item For every key $z \in A$, 
	\begin{itemize}
	\item Retrieve the set $S$ associated with key $z$. 
	\item Assign $i^* \leftarrow \min S$ and add $i^*$ to $Z$.[\footnote{The ``min" here is arbitrary; any element of $S$ will do.}]
	\item For every $i \in S$, establish the mapping $\phi(i) \mapsto i^*$.
	\end{itemize}
\item For every payoff $P_j \in \mathcal{C}'$ over with scope $V_{P_j} \subset V$, 
	\begin{itemize}
	\item Define $Z_{P_j} \doteq \{v : \exists ~ i \in V_{P_j}\text{ with } \phi(i) = v \} \subset Z$.
	\item Define a payoff $\hat{P}_j$ with scope $Z_{P_j}$ and values consistent with
		\begin{equation}
		\hat{P}_j(Z_{P_j}) = P_j\left(\phi\left(V_{P_j}\right)\right)
		\end{equation}
		where $\phi$ on a numerically indexed set $V_{P_j} \subset V$ returns an ordered multiset consistent with the indexing in $V_{P_j}$.
	\end{itemize}
\item Return a CSP $\mathcal{C}'/\phi  \doteq (Z, \{\hat{P}_j\}, W')$.
\end{itemize}
\end{algorithm}

\subsubsection{Finding $F^*$ for $\mathcal{C}'/\phi$, and Unfolding it to $F^{**}$ for $\mathcal{C}$.}

However counter-intuitive, the Variable Folding Method does call for an exact solution method for CSP's. The key is that given a CSP $\mathcal{C}$, the exact algorithm is only ever called on $\mathcal{C}'/\phi$- and such CSP's necessarily have a set of variables that is bounded by a constant. Indeed, any method may be used to solve $\mathcal{C}'/\phi$ optimally: integer programming is just as viable as dynamic programming with unlimited backtracking.

Once an optimal assignment $F^*$ is found for $\mathcal{C}'/\phi$, we need to apply it to $\mathcal{C}$. Luckily, this is extremely simple. Earlier we established that $\phi$ is simply a map from $V$ to a $Z \subset V$. From this, if we have $i \in Z$ takes value $c \in D$, simply assign this value $c$ to \textit{all} $v \in V$ that map to $i$.

\subsubsection{Practicality of The Variable Folding Method}

How can we characterize the \textit{asymptotic} size of the new variable set $Z$? Simply note that for some absolute constant $\alpha$, we have $|Z| \leq \left(\frac{\alpha}{\epsilon}\right)^{\beta q} = 2^{\psi(kq/\epsilon)}$ for some polynomial $\psi$. Of course, this is fully exponential in the ``variable" $kq/\epsilon$, but constant with respect to the number of variables in $\mathcal{C}$. For this reason, we are always ensured a folded variable set $Z$ that is bounded by some constant even for arbitrarily large $n$.

At this point we can answer a question posted at the beginning of Section \ref{sec:optRoundSchemes} : how can an algorithm be polynomial time, and ``run in time $\text{exp}(\text{exp}(\text{poly}(kq/\epsilon)))$? Of course the answer was that $k,q,$ and $\epsilon$ were not considered part of the input, but where did this strange expression come from? We showed above that the inner exponential (i.e. $\text{exp}(\text{poly}(kq/\epsilon))$) came from the asymptotic size of the set $Z$. In addition, we have now discussed how the Variable Folding Method requires that the CSP $\mathcal{C}'/\phi$ be solved \textit{exactly}. Since at the time of writing the only exact algorithms for CSP's are exponential in the number of variables, we have $\exp(|Z|) = \text{exp}(\text{exp}(\text{poly}(kq/\epsilon)))$ complexity in total.

Now some discussion on $\beta$. Of course, we are concerned with CSP instances with sufficiently large $n$ so that $\beta < nq$. Substituting the bound on $\beta$, we see that this the set of CSP instances with $ k^2q/\epsilon^3 \ll n$. For 3-SAT with $\epsilon = 0.1$, this puts us at over 500,000 variables.

At the time of writing, no exact 3-SAT solver can come close to reliably solving problems with 500,000 variables, and so for practical purposes, using an exact solver on the original CSP $\mathcal{C}$ is likely preferable to the Variable Folding Method as an approximation algorithm.

However, the Variable Folding Method can be modified and still followed ``in spirit" in ways that UGDFS cannot. Namely, rather than projecting on to a subspace of dimension $\beta$ with $\beta \gg k^2q^2/\epsilon^3$, we could project onto a more practical number of dimensions, say $3$. In such a small space, the task of explicitly constructing an $\epsilon$-net $H$ for the unit ball becomes straightforward, and the number of variables in the set $Z$ would invariably be far less than $\exp(\text{poly}(kq/\epsilon))$.

\end{document}
	
	\bibliographystyle{abbrv}
	\bibliography{references/references}
	
\end{document}

